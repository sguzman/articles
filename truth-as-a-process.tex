\documentclass{article}  
\usepackage{epsfig}  



\begin{document}  % This is where the document starts.
	
	\title{Truth as Process}   % \LaTeX makes a fancy LaTeX logo.
	\author{Salvador Guzman Jr}
	\date{\today}
	\maketitle
	\tableofcontents
	
	\section{Introduction}
	
	The Halting Problem and all the pedagogical discourse circumscribing it always struck me as a bit odd. This peculiar problem, I would introspect in private, has availed itself lavishly to the contemplative spirit of the human imagination. Through whatever inventive machinations it could afford, the broad outline of this logical exercise had found purchase in the minds of many.
	
	That this phenomena could be explained perhaps by the eagerness of computer scientists to have under their purview a body of literature rivaling that of more established intellectual traditions shines brightly for me. Should this explanation prevail, they can be forgiven for their overzealous efforts. No is immune from the lofty ambitions that one inherits when commencing an endeavor of intellectual character. Moreover, doubly can they be forgiven when their undertaking possesses the charismatic gravitas that posterity insists on bequeathing us.
	
	Truth be had, however, lest those traditions haughtily gesture beyond their worth, they suffer a similar fate. Not of course in their evocative sense of eagerness, as time for theirs has come and pass. No, no; what I mean to point out, with all this meandering word mincing is that mundane ideas are granted stature beyond what should be afforded in all sensibility to meager modicums of thought.
	
	\section{The Halting Problem Proper}
	
	So what is the problem in question that animates this article to be written? The Halting Problem, as described in my own terms, is often rendered as the proposition that is it impossible to design or engineer an algorithm, using whatever matriculation you have at your disposal, whether some arbitrarily given process (read: algorithm) will terminate.
	
	Substantive to this problem is the asymmetric exhaustion associated with the two possibilities for the process in question. Either the process exits normally and subsequently allows the algorithm to determine that the process halted or the process never halts. Note that the algorithm and the process both are assumed to run on the same thread, e.g. execution context, and must therefore trade off execution.
	
	If the process halts, it permits the algorithm the capacity to reason about the freshly terminated process and conclude that it has ended. However, the difficulty comes in the non-terminating case. Since execution context is shared, the process executing starves the algorithm of the ability not just to run, but to reason. Thusly, the algorithm never is permitted to conclude one way or another.
	
	Approximately, this resembles the dilemma. I do not wish to rehash the minutia here. Let it be said there exist many online resources that can do the problem justice. My purpose here is exposition and only weakly desire to capture the problem in its full scope.
	
	\section{Deconstructing a Problem}
	
	More paramount that I capture the problem faithfully, the general contours of the problem are what motivated me to write this article. This problem is not unique in investiture in the lapse of logical rigor to satisfy some arbitrarily given criteria. Men are creatures of innovation as much as they are of habit and on circumstance often haunted by tedium. Often their ravenous analytical ability is animus enough to bleed their contemporary tools of all their wit.
	
	\subsection{Paradox if you want it}
	What I belabor to say, with all matter of pomp and prose, is that the problem is no problem at all. What the Halting Problem, and other rigorous intellectual pursuits share, is that their inabilities to whittle away at purported intractable propositions is by design. Granted, subconscious or otherwise beholding to unconscious passions but intentional nonetheless.
	
	What always struck me as queer, whenever mathematicians (yes, they are the other guilty party I alluded to in the past) insist on the injurious and injudicious leveraging of the word paradox. I do labor as far as to permit that one man's paradox is relative to each and everyone. Nay; using the most stringent adherence to the definitions espoused wherein a paradox is concocted, one can easily arrive at any paradox proper. That is not in question and the character of the intellectually curious I do not which to mar.
	
	However, still permitted and what still persists with the tenacity that arrived at paradox in the first place is the desire to faithfully set the scene. What scene? Well, that mathematics and all of its accompanying syntax and notation are man-made, not God-given; that the tools are expressions and manifestations of human intellect; that these intellectual exercises impress more of the authors' peculiarities than of the material on which they are excercised; and finally that any expressions of ineffectual treatment of some subject or other by these authors articulates more a futility of imagination rather than a bulwark against the intellectual treatment in the first place.
	
	That is all.
	
	\subsection{Paradox, a choice?}
	
	What permits this conclusion, that paradoxes encountered amid the exercise of reason are more children of circumstance and temperament of the author than of some conclusive global statement on the nature of logic itself, is no grandiose overture. It is the simple return to vision that these are tools. Intellectual or otherwise; just tools. The constructions we use to wander our way through the hinterlands of logical consequence are no more objective, no more perfect, no more Godly, than we are. You would no more engage in the sacrilegious apotheosis of a simple hammer; wherefore then cometh your faith?
	
	\section{Plato's Revenge}
	
	If you would afford me a slight digression, I would like to speculate at length the source of this tool-worship. The name Plato should be no stranger to you. A great man of great import. What suffices for the subsequent discussion is that the man possessed a opinionated understanding of mathematics and its place in our universe.
	
	Plato envisioned that mathematical entities, whether geometric figures or numbers or proofs, were endowed with an existence that was difficult to differentiate from the nominal sense of the word as we know of it. 

\end{document}