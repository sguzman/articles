\documentclass{article}  
\usepackage{epsfig}  



\begin{document}  % This is where the document starts.
	
	\title{Truth as Process}   % \LaTeX makes a fancy LaTeX logo.
	\author{Salvador Guzman Jr}
	\date{\today}
	\maketitle
	\tableofcontents
	
	\section{Introduction}
	
	The Halting Problem and all the pedagogical discourse circumscribing it always struck me as a bit odd. This peculiar problem, I would introspect in private, has availed itself lavishly to the contemplative spirit of the human imagination. Through whatever inventive machinations it could afford, the broad outline of this logical exercise had found purchase in the minds of many.
	
	That this phenomena could be explained perhaps by the eagerness of computer scientists to have under their purview a body of literature rivaling that of more established intellectual traditions shines brightly for me. Should this explanation prevail, they can be forgiven for their overzealous efforts. No is immune from the lofty ambitions that one inherits when commencing an endeavor of intellectual character. Moreover, doubly can they be forgiven when their undertaking possesses the charismatic gravitas that posterity insists on bequeathing us.
	
	Truth be had, however, lest those traditions haughtily gesture beyond their worth, they suffer a similar fate. Not of course in their evocative sense of eagerness, as time for theirs has come and pass. No, no; what I mean to point out, with all this meandering word mincing is that mundane ideas are granted stature beyond what should be afforded in all sensibility to meager modicums of thought.
	
	\section{The Halting Problem Proper}
	
	So what is the problem in question that animates this article to be written? The Halting Problem, as described in my own terms, is often rendered as the proposition that is it impossible to design or engineer an algorithm, using whatever matriculation you have at your disposal, whether some arbitrarily given process (read: algorithm) will terminate.
	
	Substantive to this problem is the asymmetric exhaustion associated with the two possibilities for the process in question. Either the process exits normally and subsequently allows the algorithm to determine that the process halted or the process never halts. Note that the algorithm and the process both are assumed to run on the same thread, e.g. execution context, and must therefore trade off execution.
	
	If the process halts, it permits the algorithm the capacity to reason about the freshly terminated process and conclude that it has ended. However, the difficulty comes in the non-terminating case. Since execution context is shared, the process executing starves the algorithm of the ability not just to run, but to reason. Thusly, the algorithm never is permitted to conclude one way or another.
	
	Approximately, this resembles the dilemma. I do not wish to rehash the minutia here. Let it be said there exist many online resources that can do the problem justice. My purpose here is exposition and only weakly desire to capture the problem in its full scope.
	
	\section{Problem with the Problem}
	

\end{document}