\documentclass{article}  
\usepackage{epsfig}  



\begin{document}  % This is where the document starts.
	
	\title{Truth as Process}   % \LaTeX makes a fancy LaTeX logo.
	\author{Salvador Guzman Jr}
	\date{\today}
	\maketitle
	\tableofcontents
	
	\section{Introduction}
	
	The Halting Problem and all the pedagogical discourse circumscribing it always struck me as a bit odd. This peculiar problem, I would introspect in private, has availed itself lavishly to the contemplative spirit of the human imagination. Through whatever inventive machinations it could afford, the broad outline of this logical exercise had found purchase in the minds of many.
	
	That this phenomena could be explained perhaps by the eagerness of computer scientists to have under their purview a body of literature rivaling that of more established intellectual traditions shines brightly for me. Should this explanation prevail, they can be forgiven for their overzealous efforts. No is immune from the lofty ambitions that one inherits when commencing an endeavor of intellectual character. Moreover, doubly can they be forgiven when their undertaking possesses the charismatic gravitas that posterity insists on bequeathing us.
	
	Truth be had, however, lest those traditions haughtily gesture beyond their worth, they suffer a similar fate. Not of course in their evocative sense of eagerness, as time for theirs has come and pass. No, no; what I mean to point out, with all this meandering word mincing is that mundane ideas are granted stature beyond what should be afforded in all sensibility to meager modicums of thought.
	
	\section{The Halting Problem Proper}
	
	So what is the problem in question that animates this article to be written? The Halting Problem, as described in my own terms, is often rendered as the proposition that is it impossible to design or engineer an algorithm, using whatever matriculation you have at your disposal, whether some arbitrarily given process (read: algorithm) will terminate.
	
	Substantive to this problem is the asymmetric exhaustion associated with the two possibilities for the process in question. Either the process exits normally and subsequently allows the algorithm to determine that the process halted or the process never halts. Note that the algorithm and the process both are assumed to run on the same thread, e.g. execution context, and must therefore trade off execution.
	
	If the process halts, it permits the algorithm the capacity to reason about the freshly terminated process and conclude that it has ended. However, the difficulty comes in the non-terminating case. Since execution context is shared, the process executing starves the algorithm of the ability not just to run, but to reason. Thusly, the algorithm never is permitted to conclude one way or another.
	
	Approximately, this resembles the dilemma. I do not wish to rehash the minutia here. Let it be said there exist many online resources that can do the problem justice. My purpose here is exposition and only weakly desire to capture the problem in its full scope.
	
	\section{Deconstructing a Problem}
	
	More paramount that I capture the problem faithfully, the general contours of the problem are what motivated me to write this article. This problem is not unique in investiture in the lapse of logical rigor to satisfy some arbitrarily given criteria. Men are creatures of innovation as much as they are of habit and on circumstance often haunted by tedium. Often their ravenous analytical ability is animus enough to bleed their contemporary tools of all their wit.
	
	\subsection{Paradox if you want it}
	What I belabor to say, with all matter of pomp and prose, is that the problem is no problem at all. What the Halting Problem, and other rigorous intellectual pursuits share, is that their inabilities to whittle away at purported intractable propositions is by design. Granted, subconscious or otherwise beholding to unconscious passions but intentional nonetheless.
	
	What always struck me as queer, whenever mathematicians (yes, they are the other guilty party I alluded to in the past) insist on the injurious and injudicious leveraging of the word paradox. I do labor as far as to permit that one man's paradox is relative to each and everyone. Nay; using the most stringent adherence to the definitions espoused wherein a paradox is concocted, one can easily arrive at any paradox proper. That is not in question and the character of the intellectually curious I do not which to mar.
	
	However, still permitted and what still persists with the tenacity that arrived at paradox in the first place is the desire to faithfully set the scene. What scene? Well, that mathematics and all of its accompanying syntax and notation are man-made, not God-given; that the tools are expressions and manifestations of human intellect; that these intellectual exercises impress more of the authors' peculiarities than of the material on which they are excercised; and finally that any expressions of ineffectual treatment of some subject or other by these authors articulates more a futility of imagination rather than a bulwark against the intellectual treatment in the first place.
	
	That is all.
	
	\subsection{Paradox, a choice?}
	
	What permits this conclusion, that paradoxes encountered amid the exercise of reason are more children of circumstance and temperament of the author than of some conclusive global statement on the nature of logic itself, is no grandiose overture. It is the simple return to vision that these are tools. Intellectual or otherwise; just tools. The constructions we use to wander our way through the hinterlands of logical consequence are no more objective, no more perfect, no more Godly, than we are. You would no more engage in the sacrilegious apotheosis of a simple hammer; wherefore then cometh your faith?
	
	\section{Plato's Revenge}
	
	If you would afford me a slight digression, I would like to speculate at length the source of this tool-worship. The name Plato should be no stranger to you. A great man of great import. What suffices for the subsequent discussion is that the man possessed a opinionated understanding of mathematics and its place in our universe.
	
	Plato envisioned that mathematical entities, whether geometric figures or numbers or proofs, were endowed with an existence that was difficult to differentiate from the nominal sense of the word as we know of it. The eponymous platonic solids themselves existed in a divinely inspired realm of which we have scant knowledge of. Obviously, the prima facie situation is that we have knowledge enough to know of this divine realm and the constituent mathematical entities.
	
	It always striked me as odd that one would need to posit the existence of a whole inaccessible just to permit us the luxury of our mathematical fiction. Regardless, what is paramount here is the fatal entanglement of beauty and perfection that Plato assigned to mathematics. There could be little talk of math without invoking, even indirectly of its capacity to be consumed aesthetically as well.
	
	Numbers are perfect. Solids are perfect and pristine. And they all exist in some celestial realm. And we, the unfortunate residents of this miserable joyless world, can but ruminate on our fall from grace. Humans have only access to impure analogues to the perfect entities which can only guide speculation towards the more heavenly realm.
	
	This is Plato. This is his revenge. That we miserable creatures of minor fame and important and can only lust after perfection, the culmination of aesthetics, and never adequately indulge our appetite for godliness. This is his venomous tonic that humanity has only recently braved the effort to overthrow.
	
	\section{Vindictive Consequences}
	
	Given Plato, what is entailed? As aforementioned, worshiping our tools of logic is one of those consequences. That these tools, sculpted with precision and rigor, demands a captivated audience and an awed user. What follows from Plato is nothing less than tool worship. The pious fever dream that devoured us for nigh two millennia is that history has bequeathed to us numbers, solids, equations, surfaces; meticulous crafted by better men than us, men more deserving and god-like. We, the pathetic creatures that the contemporary modern age contains, should have sense enough of how unworthy we are to possess these tools. We do not deserve to behold these treasures, let alone harbor manic desires to transmogrify them for our provincial inventions. Now that we have sinned against God and coveted these gifts, preserve them for the next age. And God damn those who fail in this duty.
	
	\section{A Curse Elided}
	
	That is the curse. History has inherited us these gifts and we strive only in standing in awe of these creations.
	
	Avoiding the cult-like status that Platonism requires, it simply fails as a pragmatic paradigm. Tools are tools. As tautological as it sounds, tools have authors and authors are always of material flesh. They are designed to improve our lot in life and the authors are no more Godly than we are. Thus, what is permitted by the tacit admission of tool creation is the ability to create and modify the very same tools to our liking.
	
	That is what is meant that paradoxes are optional. It is not that they can always be avoided but rather the emphasis is on their capacity to transpire being contingent on tools.
	
	\section{Logic: Nothing but a Game}
	
	In reasoning about system of intellectual importance, it is helpful to liken the subject to something more familiar and less exotic. In my case, games are very simple to understand. They are simple in that they contain rules that are enforced but ultimately negotiable. What follows from these rules controls the resulting experience had. Logic is the same way.
	
	Logic is no different way in that its judicious exercise contains rules that are set before any logical consequences can be had. The axioms are ultimately negotiable as well.
	
	With this admission, the chief question that should occupy your mind is then what is the primary concern of logical deduction or induction when the definitions are fluid. That concern is speculation beyond what I am occupied with. What should suffice for now is that whatever the motivation is, those who play games may perchance be intimate to some fragment of wisdom in that respect.
	
	\section{What is Meant by Paradox in Common Parlance}
	
	Constituting the semantic interpretation of paradox, then, in mathematical literature is a more mundane, boring and dry definition. It is meant that given some finite set of axioms, some logical deduction is not achievable. It is agnostic on the matter on whether or not some modification of axiomatic origin would allow a concise logical denouement. 
	
	\section{Due Diligence}
	
	Yet this simple fact is presented with the same pious resolve that one voices a belief in a deity. Theistic discourse is not resented by this author at all. What I regard with derision is the ambiguity that tolerates talking of God and mathematics in the same breath. Speak of God in the firmness of God and speak of mathematics in the firmness of mathematics.
	
	If mathematics and more chiefly paradoxes were presented in the spirit of human creations to be tampered with, I imagine more progress would be had since it would not require a rebellious animus to commit sin against God to progress forward. Lowering the threshold to less sacrilegious means would be a boom to development of the mathematical and logical arts since they would possess more a material character that invites contribution.
	
	\section{Permitting a Solution to the Halting Problem}
	
	Now that all the lucid speculation is finished, I can proceed to brief outline of what I would consider a modest but satisfactory solution to the halting problem. A simple solution for the halting problem is possible if one permits the malleability of the topology of the solution. Instead of a single moment where the algorithm emits a yes or not, the concept of truth is extended to a process.
	
	Thus, a simple sketch of my solution is as follows. Each list item occurs chronologically. The algorithms is executed twice and emits two opposite results. The process in question runs in between these two calls.
	\begin{itemize}
		\item Algorithm assumes that the process will not halt, outputting no
		\item Process is ran
		\item If this step is reached, then process must have exited and algorithm outputs a yes
	\end{itemize}

	In the case that the process does not halt, the initial answer is tantamount to a conclusive and definitive answer on the matter. However, should the process actually halt, the initial answer will of course be incorrect. This should not be problem in atomic conceptions of this construction. Assuming one allows it, there are obviously available versions of this construction that posits the existence of a small time gap between each step, infinitesimally small or otherwise, would burden the entire exercise with a spurious assertion for the duration of the time gap between steps.
	
	\section{What Exactly is Supplicated}
	
	The central position of this attempted position is that the topology of the solution space should match that of the problem. To this observer, it is not judicious to stoically endure the hardship demanded by those seeking a point answer to a line problem. In short, the problem has the topology of a line since execution proceeds in a linear fashion yet a single atomic point is demanded. There is nothing brave about handicapping oneself in search of a solution. Even in the case that brevity is preferred, a characteristic of a more parsimonious approach should not wantonly burden the solution-provider with unnecessary constraints.
	
	Thus, the supplication asks fairness in dimensions. That the topology of the problem match that of the solution is prime concern. There is no reason for ask for a crippled solution, specially when pragmatically-inclined in the first place. This rough solution that I offered is better than no solution.
	
	\section{More on Truth as Process}
	More important than the attempted amelioration of the halting problem is the idea that truth can be conferred on the basis of a process. If one requires truth, it can be given on a basis that tolerates deviation from true veracity as long as the deviation is within some agreed degree. One thinks of probability theory and statistics which has the idea of a confidence level. There is no reason that truth cannot function on a more loose footing to allow for more flexibility.
	
	In distributed systems in computing, there is a well-known theorem called the CAP theorem that dictates the trade-off inherent in a distributed domain. In such a setting, truth as a process that eventually is correct is not such an alien concept.
	
	It is my humble opinion that mathematicians and logicians can learn to treat truth with more malleability to allow for the formation of more formidable domain tools.

\end{document}