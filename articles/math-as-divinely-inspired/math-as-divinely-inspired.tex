\documentclass{article}  
\usepackage[utf8]{inputenc}
\usepackage{amsmath}
\usepackage{fancyhdr}
\usepackage[english]{babel}

\setlength{\parindent}{4em}
\renewcommand{\baselinestretch}{1.5}


\begin{document}
	
	\title{Math as Divinely Inspired}
	\author{Salvador Guzm\'{a}n Jr}
	\date{\today}
	\maketitle
	\tableofcontents
	
	\section{Introduction}
	The extraordinary purchase that the intellectual blossoming mathematics experienced is beyond reproach. This certain scholarly field, as an academic discipline and as a practical application to engineering problems, is the quintessential success story for why raw intellectualism delivers on its promises. To say that this field found success with even the more prudishly pragmatic elements of society is to deprive mathematics of its many splendid laurels that intend to pay dividends commensurate with the prosperity of humanity. In deed, it is our primary intellectual beast of burden on which the modern world rests.
	
	And to say that the mathematical edifice bequeathed to us by the historical process captured our effervescent spirit is to mitigate our immodest revel in the art. Surely, we have been enraptured by this craft and the practitioners thereof have experienced a lofty elevation in status among the common populace. This is a very interesting development for a field known to be arcane and relatively inaccessible to the wider audience. One imagines it is this very unapproachability that contributes the consensus of our age and conclusively regards math as a divinely inspired creation.
	
	\section{Discursive Realities - What is Math?}
	The extent of the cognizance that most of our citizenry has in respect to math is circumscribed by their grade school experience. The math curriculum of our today has gone through an elaborate gauntlet of vicissitudes. Current mathematical pedagogy can best be understand as constituted by a series of fractious schools of thoughts that emphasize difference aspects of the mathematical arts.
	
	\section{Mathematics as Prophetic Revelation}
	
	\section{The Holy Sacraments of Mathematics}
	\subsection{Sacred Logic}
	\subsection{Apotheosis of Symbols}
	\subsection{Mathematics as a Social Practice}
	\subsection{Mathematics as a Teleological Undertaking}
	\subsection{The Sublime Truth in Mathematics}
	\subsection{Mathematics's Expansive Metaphysical Empire}
	\subsection{Semiotics of Another Holy Priesthood}
	\section{Deconstructing the Sacraments}
	\subsection{Logic as Cause and Effect}
	
	

\end{document}