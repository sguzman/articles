\documentclass{article}  
\usepackage[utf8]{inputenc}
\usepackage{amsmath}
\usepackage{fancyhdr}
\usepackage[english]{babel}

\setlength{\parindent}{4em}
\renewcommand{\baselinestretch}{1.5}


\begin{document}
	
	\title{\Large{\textbf{Math as Divinely Inspired}}}
	
	\author{Salvador Guzm\'{a}n Jr}
	\date{\today}
	\maketitle
	
	\pagebreak
	\pagebreak
	\tableofcontents
	\pagebreak
	
	\section{Introduction}
	The extraordinary purchase that the intellectual blossoming mathematics experienced is beyond reproach. This certain scholarly field, as an academic discipline and as a practical application to engineering problems, is the quintessential success story for why raw intellectualism delivers on its promises. To say that this field found success with even the more prudishly pragmatic elements of society is to deprive mathematics of its many more splendid laurels that intend to pay dividends commensurate with the prosperity of humanity. Indeed, it is our primary intellectual beast of burden on which the modern world rests.
	
	And to say that the mathematical edifice bequeathed to us by the historical process captured our effervescent spirit is to mitigate our immodest revel in the art. Surely, we have been enraptured by this craft and the practitioners thereof have experienced a lofty elevation in status among the common populace. This is a very interesting development for a field known to be arcane and relatively inaccessible to the wider audience. One imagines it is this very unapproachability that contributes the consensus of our age and conclusively regards math as a divinely inspired creation.
	
	And it is this consensus, that mathematics has evaded the perdition that haunts humanity since the fall; that this craft offers us a glimpse at the heavenly realms that elude our vision; and that there is no incremental historical development to speak of when discussing this body work, as such a divine creation would not allow itself to be mended piecewise. Yes it is this consensus that I wish to disabuse.
	
	\section{Discursive Realities - What is Math?}
	The extent permitted by the cognizance that most of our citizenry has in respect to math is circumscribed by their grade school experience. The math curriculum of today has gone through an elaborate gauntlet of vicissitudes. Current mathematical pedagogy can best be understood as constituted by a series of fractious schools of thought that purport to thoroughly address the concerns of each other and evince themselves as possessing the correct approach to pedagogy. Iterate this scenario for decades and we arrive at our current curriculum.
	
	Far be it from me, someone particular nondescript affiliation with those formal pedagogical institutions to comment on the 
	
	\section{Mathematics as Prophetic Revelation}
	
	\section{The Holy Sacraments of Mathematics}
	\subsection{Sacred Logic}
	\subsection{Apotheosis of Symbols}
	\subsection{Mathematics as a Social Practice}
	\subsection{Mathematics as a Teleological Undertaking}
	\subsection{The Sublime Truth in Mathematics}
	\subsection{Mathematics's Expansive Metaphysical Empire}
	\subsection{Semiotics of Another Holy Priesthood}
	\section{Deconstructing the Sacraments}
	\subsection{Logic as Cause and Effect}
	
	

\end{document}