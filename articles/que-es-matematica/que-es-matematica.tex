\documentclass{article}  
\usepackage[utf8]{inputenc}
\usepackage{amsmath}
\usepackage{esint}
\usepackage{fancyhdr}
\usepackage[spanish]{babel}
\usepackage{epigraph}
\usepackage{beton}
\usepackage{euler}
\usepackage[OT1]{fontenc}
\usepackage{yfonts}

\setlength{\parindent}{4em}
\renewcommand{\baselinestretch}{1.5}

\begin{document}
	\selectlanguage{spanish}
	\title{\Huge{\textswab{Que es Matem\'{a}tica}}}
	\author{Salvador Guzm\'{a}n Jr}
	\date{\today}
	\maketitle
	
	\pagebreak
	\begin{center}
		Por mi diosito, mi lindo M\'{e}xico y mi querida madre
		\linebreak
		Que siguen prosperando los tres
	\end{center}
	\pagebreak
	\begin{center}
		\epigraph{
			No est\'{a} muerto lo que puede acostarse eternamente
			y con e\'{o}nes extraños tambi\'{e}n la muerte
			puede morir
		}{\textit{
			HP Lovecraft
		}}
		\epigraph{
			\begin{math}
				\int_{\partial\Omega} \omega=\int_{\Omega} \emph{d}\omega
			\end{math}
			\linebreak
			\linebreak
			\begin{math}
				\iint_{\emph{S}}(\vec{\nabla}\times\vec{B})\cdot \emph{d}\vec{\emph{S}}=\oint_{\emph{C}}\vec{B}\cdot d \vec{l}
			\end{math}
		}{
			\textit{Teorema de Stokes}
		}

		\epigraph{
			Aunque ande en valle de sombra de muerte,
			No temer\'{e} mal alguno, porque t\'{u} estarás conmigo;
			Tu vara y tu cayado me infundir\'{a}n aliento
		}{
			\textit{Salmos 23:4}
		}
	\end{center}
	\pagebreak
	\tableofcontents
	\pagebreak
	
	\section{\Large{\fontfamily{phv}\selectfont{\textsc{Pr\'{o}logo}}}}
	\subsection{\fontfamily{phv}\selectfont{\textsc{Estas Lagrimas son para Usted}}}
	Querido lector ocioso, cuente me en que manera encontr\'{o} este manuscrito. De verdad, \mbox{?`}conoce de que se trata estas palabras? Perd\'{o}n lector perdido; tener una vida propia en que se encuentra leyendo este art\'{i}culo debe ser muy difícil. P\'{o}bre lector ocioso; debe saber que no ay nada que no hago por usted para evitar consecuencias suya; de estar leyendo palabras tan p\'{o}bes. Con toda la lastima, entonces, me da pena decirle que usted contesto la llamada de la sirena ahogada y esta invitado a sufrir con migo!


	\mbox{?`}De que se tratan estas palabras?, usted se supone me pregunta. Querido, es lamentable que si sabia yo, nunca lo hubiera intentado a escribir este usura delincuente. Lo que intento a demonstrar a usted y cualquier alma perdida que estas palabras alcancen, con o en contra de su voluntad, es mi desconocimiento contextual. Bueno, pregunta usted, \mbox{?`}a que refiere mi ignorancia? \mbox{?`}Que delito me a visitado que me encuentra sin conocimiento? Disculpe; si sabia yo, no le preguntaba. De verdad la cosa es a regresar conocimiento que era millo. Y no pienso que por estando haraganeando me perd\'{i} cosa preciosa.
	
	\mbox{!`}No piense cosa tan malvada de mi, lector! Me hace da\~{n}o imputar prop\'{o}tos
	
	No, lector. Lo que pertenece a una persona es algo inapreciable.
	
	
	
\end{document}