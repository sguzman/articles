\documentclass{article}  
\usepackage[utf8]{inputenc}
\usepackage{amsmath}
\usepackage{esint}
\usepackage{fancyhdr}
\usepackage[spanish]{babel}
\usepackage{epigraph}

\setlength{\parindent}{4em}
\renewcommand{\baselinestretch}{1.5}

\begin{document}
	\selectlanguage{spanish}
	\title{\Large{\textbf{¿Que es la matemática?}}}
	
	\author{Salvador Guzm\'{a}n Jr}
	\date{\today}
	\maketitle
	
	\pagebreak
	\begin{center}
		\thispagestyle{empty}
		\vspace*{\fill}
		\textbf{Por mi diosito, mi lindo M\'{e}xico y mi querida} madre
		\linebreak
		\textbf{Que siguen prosperando los tres}
		\vspace*{\fill}
	\end{center}
	\pagebreak
	\begin{center}
		\epigraph{
			No est\'{a} muerto lo que puede acostarse eternamente
			y con eones extraños  incluso la muerte puede morir
		}{\textit{
			HP Lovecraft
		}}
		\epigraph{
			\begin{math}
				\int_{\partial\Omega} \omega=\int_{\Omega} \emph{d}\omega
			\end{math}
		\linebreak
		\linebreak
			\begin{math}
				\iint_{\emph{S}}(\vec{\nabla}\times\vec{B})\cdot \emph{d}\vec{\emph{S}}=\oint_{\emph{C}}\vec{B}\cdot d \vec{l}
			\end{math}
		}{
			Teorema de Stokes
		}
	\end{center}
	\pagebreak
	\tableofcontents
	\pagebreak
	
	\section{Prólogo}
\end{document}