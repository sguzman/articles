\documentclass{article}  
\usepackage[utf8]{inputenc}
\usepackage{amsmath}
\usepackage{esint}
\usepackage{fancyhdr}
\usepackage[spanish]{babel}
\usepackage{epigraph}
\usepackage{beton}
\usepackage{euler}
\usepackage[OT1]{fontenc}

\setlength{\parindent}{4em}
\renewcommand{\baselinestretch}{1.5}

\begin{document}
	\selectlanguage{spanish}
	\title{
		\Large{
				\fontfamily{phv}
				\selectfont{
					¿\textbf{Q}\textsc{ue es la}
					\textbf{M}\textsc{atem\'{a}tica}?
				}
		}
	}
	
	\author{Salvador Guzm\'{a}n Jr}
	\date{\today}
	\maketitle
	
	\pagebreak
	\begin{center}
		\thispagestyle{empty}
		\vspace*{\fill}
					\textbf{
			\fontfamily{phv}
			\selectfont{
				Por mi diosito, mi lindo M\'{e}xico y mi querida madre
				\linebreak
				Que siguen prosperando los tres
			}
		}
		
		\vspace*{\fill}
	\end{center}
	\pagebreak
	\begin{center}
		\epigraph{
			No est\'{a} muerto lo que puede acostarse eternamente
			y con eones extraños  incluso la muerte
			puede morir
		}{\textit{
			HP Lovecraft
		}}
		\epigraph{
			\begin{math}
				\int_{\partial\Omega} \omega=\int_{\Omega} \emph{d}\omega
			\end{math}
			\linebreak
			\linebreak
			\begin{math}
				\iint_{\emph{S}}(\vec{\nabla}\times\vec{B})\cdot \emph{d}\vec{\emph{S}}=\oint_{\emph{C}}\vec{B}\cdot d \vec{l}
			\end{math}
		}{
			\textit{Teorema de Stokes}
		}

		\epigraph{
			Aunque ande en valle de sombra de muerte,
			No temeré mal alguno, porque tú estarás conmigo;
			Tu vara y tu cayado me infundirán aliento
		}{
			\textit{Salmos 23:4}
		}
	\end{center}
	\pagebreak
	\tableofcontents
	\pagebreak
	
	\section{Prólogo}
	Querido lector ocioso,
\end{document}