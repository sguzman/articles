\documentclass[12pt]{article}

\usepackage{amsmath}
\usepackage{amssymb}
\usepackage{amsthm}
\usepackage{graphicx}
\usepackage{hyperref}
\usepackage{color}
\usepackage{enumerate}

\begin{document}
\title{What is a book?\\
\large Fungibility of Books and Their Composite Words}
\newpage 

\author{Salvador Guzman}
\date{\today}
\newpage

\maketitle
\newpage

\begin{abstract}
Books capture the essence of knowledge distilled into a verbal consolidation. In
this sense the physical medium that constitutes the books proper is not what
compels attention. What truly continues to rapture our imagination is the
stability of the written word. Written documents have proven that communication
can be stored, transmitted, reproduced and expressed in a variety of ways. All
wealth of actions that the written word commands can be done all without every
modifying the original intent. This is the essence of the word and the
books that house them. This can only be done by accepting certain fungible
facets of words, books and knowledge. A word is a word and knowledge can be
built incrementally without regard to any purported \textbf{sine qua non} that may
lurk in the shadows. In this document, I make the case for the fungibility of
the written word.
\end{abstract}

\newpage

\tableofcontents
\newpage

\section{Introduction}
The story of the word is written long and deep in the history of mankind. Our
fascination with what is written demands solid analysis of its causes and its
consequences. For now, I seek to liberate the context of what is at stake. While I
demure from comprehensive deliberation, I will attempt to establish common cause
with the earnest reader. My thesis statement is that books are fungible. This
hypothesis relies on the premise of the cultural character of the written word.
The best that an author can do is point towards objective truths beyond 
the shores of their time and place. Their epistemological island is of course an
invitation to the reader to explore what the writer could not. The language the
author employs, with its compositional lexical, syntax and semantic
rapprochement are an attempt to scaffold at truths with provincial tools. Thus,
what is necessarily permitted is the provincial attempt at the universal.

\section{The Nature of Books}
What is a book? That is what I seek to answer. In this endeavor, what I must
maintain is that the lust for words not blind us to the constituent nature of
the written word. To accentuate the point, one can imagine what is required when
composing a book. In this particular exercise, I care not what the book is
physically composed of. When I refer to book, I mean the total sum of the
written verbal content in the book proper. It is this verbal book that I
examine. I will make no more distinction between the physical book and the
verbal book albeit this one: the verbal book is the muse.

\subsection{Typed Tuple}
In programming, the concept of a tuple is common place. It is merely a list of
categories of things. For example, if one were to describe a specific car, one
would use a make, model and year. These three qualities would be the tuple to
describe the car. There are many different ways to formulate a tuple in the
designation of an object. The impetus for this sort of formal rigor is to allow
unambiguous identification of objects. This would lead trivially to algebraic and
further formal analysis.

The type of tuple is the cartesian product of the constituent categories. In the
above example, the formalism would yield,

$Cars := (Make \cdot Model \cdot Year)$. 

There should be nothing impressive about this. What is significant is the
simplicity. One imagines that the goal of this formalism is the ability to
specify and decompose more complicated objects into simpler ones. Being a bit
more imaginative, one could invoke the notion of a vector space and the specfic
car as a linear combination of the constituent categories. Since by the nature
of being different types, the resulting combination should yield immiscible
terms and thus do the categories maintain their identity.

$ Car_i := \alpha_i \cdot x_{make} + \beta_i \cdot x_{model} + \gamma_i \cdot x_{year} $ 

where $x_{make}$, $x_{model}$ and $x_{year}$ are the respective categories and
the scalar coefficients $\alpha_i$, $\beta_i$ and $\gamma_i$ are the scalar
values that choose $x_{make}$, $x_{model}$ and $x_{year}$ respectively. The But this is not a book
about tuples. The goal is their use as formal descriptors of an object.

In the above example, we can use the tuple to identify a specific car. Of
course, this tuple is anemic in that it does not exhaustive describe all
possible cars. Not only are there other categories that would describe vehicles
more aptly, but even within a concrete tuple, there are multiple instantiations
allowed. So this attempt must be tempered by what is feasible and more
importantly by what is desired. One could take this discussion of tuples in any
direction they want. For example, I did not even mention that tuples themselves
could easily be nested inside of another tuple. That is because my particular
use for tuples is relative simple and straightforward. What follows next is a model attempt to
differentiate books by a tuple. 

\subsection{Unique Identity of Books by Way of Typed Tuple}
A possible formulation of a tuple for books is,

$Books := (Author \cdot Title \cdot Edition)$.

This is a simple tuple that describes a book in terms of the properties I find
significant. The author, title and edition, I maintain possess a unique
identity. One could allow for other categories and thus this attempt would be
deficient in this regard. A publisher could have different expectations of the
book and change small attributes. In this formulation, there is not even a
reference to the language of the book, even though it is language that inspired
this composition in the first place. 

So what is meant by this? The intention, my intention, is that these additional
categories of description do not inspire an alternative intention by the author.
If the same book were written by the same author in two different languages, I
would insist that they are the same book. The language is not the book. I claim
that the intention of the author, captured by the tuple, is the book. Of course,
I must claim this in a less formal and more ad hoc manner. I have not rigorous
constructed such a book. And I refuse to. Having promulgated the formality of
the idea, now I will demure and speak in more casual language.

\subsection{The Dead Hand of Formality}
This jettison of matters formal is no innovation on my part. All that is meant
is that I do not possess the proper formal semantics to encode the construction
of the identity of books. Perchance, I will be my serendipity to casually come
across the lumbering beast, or perhaps build it myself. For now, I resign from
an informal description of the thing, the book, the word, since I can accomplish
more by this occasion. I will leave the formalism of the identity of the book to
future adventurers. I am too beside myself with languor to attempt such a feat now. 

\section{The Nature of Words}
In the previous exercise, I demonstrated that the identity of the book is
commensurate with a tuple. Not necessarily my tuple but describable by one.
However, what even this venture into rigor fails is that more important than the
tuple is the intention of author. I claim that ultimately the identity of the
book is the identity of the author's intention. What is the intention? What does
the author intend? This is more important than the tuple. Perhaps a finer
modeling of the tuple would allow for dictate by intention. Perhaps not. All
that is needed to continue is that the tuple I created, that I jovially and
eagerly hope would constitute a valid and unambiguous construction, fails. I
believe that maybe a tuple with more finesse and a more apt panoply of
categories to draw from could improve on my rough draft. That is not relevant.
What is that ultimately material is that we will be modeling the intention of
the author. Thus, the nature of words is the author's intention.

\subsection{What We Mean By Words}
In the same way that Foucault sought to extricate the word from the identity of
the author, we can go far in our analysis of the identity of a book by
submitting as much. While the preceding paragraphs invited the reader to think
of a book as being the consequence of the author's identity, the flow of the
unilateral conversation can be reversed. In fact, I would prefer to see identity
as a downstream consequence of the words themselves. At the risk of rendering my
formal shenanigans a vice, I prefer the words foremost speak for themselves.
This reflection of the direction of analysis I think is more conducive to my
idiosyncratic style of thinking. The simple building blocks yield more profound
creatures of insight. A small tangent aside, if words allow for the
manifestation of more complex objects, then what is instigated is that the
stream of consciousness that pervades minds is the ultimate source of identity.
The inner monologue of the mind is commensurate with identity. At any rate, it is
the words that we mean.

\section{Putting the Author on Their Head}
If the words are a specie of identity, then words become the building blocks of
identificatious objects. Even if I were to decouple the two conflated
definitions of the word identity, formal algebraic identity and identity of the
author, the conclusion would still hold. Thus, do not let my banal games with
word sore you on the word as identity. The notion continues to command the
lion's share of primary focus in this discussion. By now, it should be ensconced
in the mind of the reader the path that is being tread towards the goal of
sublimation of all to the written word. It is a formidable creation and we
should behold it for the majesty that it is. 

That said, we should recite how much we have funambulated. The treacherous
tight-rope path we have forged together is a testament to the strength of the
written word. We started with looking at the identity of a book. We then sought
to strengthen and produce a toolkit to describe the identity of a book. Next we
put the author on their head and made their identity and explicit intention
second to the written material they produced. This makes words the foremost
phenomena in this analysis. Where I want to arrive now is that it is the words
that achieve the identity of the book. The author and their intention is
secondary to this affect.

Now that we have our petulant little words, how do we achieve immortality of the
spirit? Or rather, this bottom up approach to the study of words, we will find,
is more conducive to the study of the nature of words. The words themselves the
primal experience; all else is secondary. The words are the primary phenomena in
question and in answer. The question this tract strategic analysis engenders is
what advantage does this confer on our efforts? I would not be so presumptuous
as to suggest this course of action without much thought dedicated to the consequences.

What we gain is the world. Assigning words as the primary focus of our
investigation, we develop a more functional model of semantics. When I say
functional, I mean that the analysis of the word composite lends itself more to
a viable approach of structuring in an unambiguous manner the projection of the
semantic space generated by the words. I wish I could say more about semantics,
but I have already drawn this conversation long by this tedious diversion.

\section{Ambiguity, the Semantic Devil}
With this new approach to the study of words, we can now begin to address the
space of meaning. The space of meaning is the space of all possible meanings.
With the addition of every word, the space of semantics is narrowed. The
eventual meaning that is arrived by the text is then attributed to the author.
Thus, in this way, there can be an isomorphism between this new approach and
"author's intention" approach. All this is in service of mitigating ambiguity.
Any and all attempts at formal analysis must eventually confront how ambiguity
bedevils and jeopardizes the validity of the enterprise. Truly, ambiguity is our
own dear devil. All efforts in clarity and transparency must eventually develop
their own or bootstrap formidable tools to combat the ambiguity and minimize it
where ever they can.

\subsection{Formal Ambiguity}
While I don't wish to belabor further this document (I am becoming bored of this
topic at present), I want to speak speculatively about the nature of ambiguity.
One would wish that we possessed a formal semantics to describe the nature of
ambiguity. While there have attempts to define what is meant (or not meant) by
ambiguity, they have faltered on domain specific shores. What I am asking for is
a more general \textit{ambiguity theory}. Dedicating an entire realm of
literature to ambiguity will help us uncover the shared structure of ambiguity
across different domains. I pray one sees how asking structure out of ambiguity
is not any oxymoron. Maybe I will be the author. Or maybe you, dear reader.

\section{Semantics as the Fungible Scaffold}
Semantics possesses a composite character. It is built upt incrementally by the
addition lexical items and syntactic groups. It is in the particular assortment
of these items that the meaning of the text is derived in a bottom up fashion.
It can only be this way because the meaning of the text is the meaning of the
words. As you may understand now, text is just shorthand for a collection of
words. Given the composite nature of this enterprise of meaning, one can see how
semantics scaffolds on itself. Thus, the final text is but a cumulative product
of sequence of lexemes, collected into further syntactic groups.

\section{Means by Any Other Mean}
The goal of any document or any treatise is to convey some eternal truth. At
least, that is what ideally texts should be concerned with. While I realize that
not all written material is concerned with objectivity, I am not practically
concerned with these provincial works. Authors should write words and words
should reach for the eternal. Looking at the converse, why should I concern
myself with reading works that are not concerned with truth? I will leave it to
my intellectual descendants to prove otherwise. Ah! the beauty of progenitors.

\subsection{Fermented Lexemes}
Given that words suggests a relationship to eternal truths, we are now in an
envious position to conclusive define the entire enterprise of words. Authors
write to tell. Their words aim at the eternal through provisionally provincial
means. The contemporaneous verbal expressions permeate the work. While the aim
of the work remains true across centuries, the age of the rhetoric and other verbal
devices starts to show. What is to be understood is the text at the time of
writing. We can pin this semantic interpretation of the text by understanding
the author's intention and the words. By pinning down these two categories with
the text, the meaning of the text is forever fixed and accessible beyond the
narrow bounds of culture and time.

\subsection{Functional Space of Truth}
This gambit is support by the relative size of the functional space of truth.
Eternal, read objective, truths are few and far between. However, the space of
provincialism is much larger. That is because the latter contains snippets of
the truth unfully realized. It is not quite a distortion of the truth but more
of romance of whatever partial truth is possessed. This ascetic capture of the
eternal invites a psychological assessment of why humans become enamored with
their snippet of the truth. I will leave this for another time.

What the above analysis suggests is that the space of truth is much smaller than
attempts at the truth. If someone is trying at some truth, they are aiming at
the same thing. One can imagine a metric space that collapses the distance of
attempts at the truth towards their eternal goals. In this way, there is no
formal distinction between attempts. And it is recognizing this fact that we can
begin to rigorously declare a sort of algebra of truth.

\section{Towards an Abelian Algebra of Epistemology}
The main ally of ambiguity is the reliance on order to dictate formal semantics.
One can imagine a formal semantics that is agnostic to the order of the words.
What I am saying has already been done for more domain specific domains. In
mathematics, a commutative algebra is one that is agnostic to the order of the
terms. Computation of any of the incumbent operators does not demand any
particular order. I want that for truth. Truth should not depend on order. The
only way to achieve this is by enriching our verbs or nouns or both. One can
imagine \textit{truth} operators that lust after truth; one can also imagine
\textit{truth} terms that contain the attempt.

In this model, truth is understood as arrived at or not. This distinction
relies on the semantic purview of sets from sets theory rather than sentences
with their order-dependent reveal. This can be accomplished as mentioned in the
previous paragraph by enriching our operational vocabulary of language. Formal
semantics that remains immiscible under duplicates and faithful to attempts is
the ideal mechanism to preserve commutativity. This is the algebra of truth. One
that is not concordant and limited by the order of words. One can think of in
comparison an analytical language and one that relies more on inflectional
morphology. The latter relies heavily on enriched objects. Maybe someday we will
have to pay the sins of pushing the pivots of syntax onto the lexicon. But today
is not that day.

\end{document}