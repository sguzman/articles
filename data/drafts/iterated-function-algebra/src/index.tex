\maketitle
\pagebreak

\begin{abstract}
    One of the greatest intellectual crimes to beset us in the 20th-century was
    the premature death of the formalist program. The millennia old dream of
    \textit{solving math} was never realized as our efforts fell short of our
    ambition. David Hilbert, with all his grace, his effulgent brilliance, his
    professional magnanimity, and his unflinching dedication, was left with only
    disgruntled disappointment. The formalist program was a noble effort, held
    aloft by the unyielding conviction and charisma of the foremost
    mathematicians of the age. This forlorn vignette is rendered at least
    somewhat emotionally digestible by the developments that followed. The dream
    of \textbf{syntax is all} became partially realized by the contributions of Haskell
    Curry, Alonzo Church, Stephen Kleene, Moses Schönfinkel and others. With
    their syntactic approach to mathematics, we capture the compositional beauty
    of infinity in our humble symbol. And thus, we compile the syntactic face of God.
\end{abstract}
\pagebreak
\tableofcontents
\pagebreak

\section{The Tragedy of Formalism}
The death of the formalism program was a necessary tragedy for the advancement
of mathematics. Gone were the ensconced notions were the naive intuitions of
mathematical computation. Rather, it was the lack of such a computational
foundation for mathematics that needed to be addressed. It was around this time
that mathematicians began to rigorously define what it means for a statement to
be provable. Without a computational foundation, the notion of provability
rested on a tentative and ambiguous folklore. For this heroic death, we
inherited a proliferation of computable logical systems that could be used to
encode mathematical ideas and theorems. The program's death died for our crass
naivite and ensured the salvation of mathematics for the modern age. It is here
where we demark the beginning of computation.

\subsection{Poetry of Computation}
The poetry of computation is the poetry of the formalist program reborn.
Whether it be lambda calculus or combinatory logic, the accentuation of syntax
is what deliberately marks this change in tradition. What Frege set in motion
is that the story of math and logic is in part the story of syntax. And it is
with syntax that we seek to capture the semantics of mathematics. Whether we
behold the history of mathematical symbols establishing what we associate with
the mathematical arts, or modern compilers that bleed out the captured
semantics of a language to a hardware equivalent, syntax is the poetry of
computation. What the formalist program lacked in ultimate success, it made up
for in intuition and beauty.

\subsection{A Nightmare Embraced}
Of course, the program is not without its thorns. To assume that the program
must be a feasible one is to posit that mathematics is devoid of content beyond
its superficial and meaningless syntax. The inevitable progression of this
species of thought is that mathematics is a mere syntactic game. Any semantic
model we attach to this game must submit to the reality of being mere fiction.
And if this holds, then our notions of mathematics fall into the void of
nihilism and thus embrace the nightmare of extinction. This nightmare becomes a
wonderful dream only when coupled with a more human focused approach to
mathematics, such as intuitionism. I won't elaborate on what this latter
program entails suffice to say that it dulls the sting of the void. Once we
compliment our syntax program with a phenomenological narcotic, we can
commandeer the development of mathematics where the syntax demands and inject
our semantic creature comforts as an afterthought. After all, mathematics is a
inductive, deductive and analytical art; there is no room for a bleeding heart.

\subsection{A Gentle Promenade Down The Road to Perdition}
Let us follow this dreamy reverie down the whirlpool of damnation. Retaining
the insight that our formalist program may not condemn us to the void by a
serendipitous pairing to some phenomenological strand of thought. I am not
concerned what flavor or brand of special pleading is provisioned. Consider the
program plead. Now, envision if you might that the program were not doomed to
the ash bin of history. In this alternative hypothetical universe, the program
succeeds. We have successfully reduced all of mathematics up to that point to
pure symbolic function of syntax. Semantics thus submits readily to capture by
the mystical symbol. The program is a success. An element of this universe is
in substance some ready made mathematical object that computes algebra and
reduces it to theorems of our liking. Our appetite for conclusive proofs and
pure deductive reasoning is sated. That is the dream. Well, if I have not
belabored enough (forgive me!), I modestly posit that claims of the death of
this program are of mild exaggeration. The program is not dead, it is merely
gone dormant while our syntactic tools sharpened.

\section{Building A Sea-fairing Vessel out of a Corpse}
Dear reader, may you forgive me for my crass verbosity. I fear it is the only
way my sinister little thoughts can find their way to a wider audience.
Consider it a concession well spent. I do indeed hold that the purview of the
formalist institution continues full steam, unencumbered by the romance of
larger narratives. Again, I outlined in cursory detail the different syntactic
program that explored what is possible with theory. As maligned as syntax is as
a foundation for mathematics, it has not heeded the call to abolish itself out
of existence. Progress, once tasted, is not easily relinquished. What has been
thwarted by historical account conclusively is the naive version of the
program. By necessity, any research that follows in the aftermath must deviate
in substance from the initial thrust of the program. The impetus of critique
imparts a degree of resiliency to subsequent efforts. And it is these efforts
that carry the banner of formalism, notwithstanding the more judicious
features.

\subsection{The Naive Intuition of the Initial Program}
Why did the program fail? Did it succumb to the bleeding that sparring with
intuitionism? Probably not as both programs are rather dead in their initial
manifestation. I hold the notion that the intuition behind each of these
efforts complimented each other pleasantly and should not have been competitors
for our hearts and minds. But such is life when a post-mortem autopsy yields
two corpses instead one. It is an incident of history I feel that David Hilbert
triumphed and that he dueled the other school in the first place. The syntactic
approach of Hilbert was a more pragmatic one. It focused on the material
conditions in which mathematics was practiced and developed. Syntax was the
conduit for his program because syntax was what conveyed the semantic
information of mathematics. To Hilbert, analysis of the syntax, and thus the
grammar of mathematics, was the key to understanding the exposition of a
computable basis. It is not to say that Hilbert believed mathematics to be the
mindless churning of symbols and syntax was the heart of mathematics. But in
the pliant concession of this profile of the art do we extract the possible
essence of a feasible program. In the rotten carcass of its demise, do we find
brilliant gems of insight that can be salvaged for future use.

\subsection{Rescinding The Full-throated Ambition of Formalism}
Doomed ambition need not be a death sentence and even carrion can provide a
cozy abode if for maggots alone. What precluded the longevity of the enterprise
is the pallid condition of research into syntax and formal languages.
Contemporaries would have to wait several decades until the illustrious formal
studies of grammars and languages produced more solid foundations. Hilbert
himself must have known this to some degree. His questions had a rather
informal air to them and left as an open question whether new mathematical
machinery would be needed to satisfy his demands. Given how much research was
put into syntax, it would be a daunting task if he knew how much literature it
would spawn. To wit: lambda calculus, combinatory logic, type theory, category
theory and so on. These are a few syntax-related or algebraic-focused
approaches to studying mathematics. The list is not exhaustive and the future
is sure to spawn more. Thus, it is here that we find our departure of his
program, respectfully pay homage to this effort, offer eulogies and then move
on to greener pastures.

\subsection{Seduction of Intuitionism}
While ultimately vindicated by the promulgation of later research, intuitionism
was dealt quite a fatal blow. Hilbert's words spoken against intuitionism
proved potent vigour against it. His champion of "more results", not less, was
the rhetoric of the day. I hope in our age, we see the flaws in such reasoning.
Intuitionism hoped to bring mathematics to us and create a more constructive
substrate to leverage for the communicating of mathematical reasoning. I admit
to prejudice hearing it from Hilbert himself and saw myself as opposed out of
principal such aspirational goals. But merit of extinguishing the practice of
mathematicians, as practiced for millennia, was at the time not compelling.
Strange it is today, to this humble observer, that the argument of ipso facto
posteriority would rule the day in discussion of advancing technology. Do we
say today that new programming languages should not be fashioned to not
eclipsed the interest of previous languages and programmers? Outrages! But such
was the time and such was curling of the monkey paw dealing intuitionism a
dishonorable discharge from the ranks of mathematics.

\section{Ambitions Laid Bare}
So what is it that I seek? Thus far we have arrived at two dead ends and no
light to elucidate our dark path. Have we fully exhausted our resolve? Of
course not. What we are now left is the raison d'etre of this document. With
everything I have said in heart and also with the motivations contained in the
initial undertakings, I now present my own ambitions for the future of
mathematics. I want a syntactic basis for mathematics. I want this to reduce
the development of this field as a matter of logical consequence extrapolated
from the choice of syntax. I also believe wholeheartedly the intuitionism
utopia of a constructive and human-digestible vantage point for tackling
analysis. It is in the spiritual synthesis in these things that some modicum of
genius is found. But I will leave that enterprise for some future date. Suffice
for now is my own modest ambitions.

\subsection{Princess in the Tower of Algebra}
It is my strong conviction that algebra as it is understood in mathematics is
substitute for other concepts. Algebra as a principal captures structure,
symmetry and dynamics of a system of objects. It is readily amenable to
analysis by syntax since formalism is indeed syntax itself. It is very tempting
to transgress the boundary of this observation and claim that algebra is only
syntax. I will leave that transgression for some other day. For now, let's
assume that algebra and algebras are pliant things and syntax is no censurious
approach to understanding them. What I am wrestling with, dear reader, is
generalizing the notion of algebra to a more general notion of structure
permitting computation. We have an algebra of our beloved objects and can with
exceeding ease compute properties with these systems. What if we had a algebra
for everything? What if we had an algebra for generating other algebras? That
licentious thought must sadly be put to rest for now. But I hope the temptation
of such a beauty will stay with you and haunt your languid summer days.

\subsection{Outline of my Ambitions}
My train of thought lead me to extrapolate from syntactic principals to more
general laws governing the dynamics of algebraic objects. I am effervescent
with excitement to share what I have found! My personal research had found
purchase in implementing syntactic methods for polynomials. Predilection proper
of mine lead me to pursue a novel approach to computing the closed form of
iterated polynomials. Knowing the particular polynomial function from the start
yielding the closed form. In outline and part, here follows my general line of
reasoning.

\begin{enumerate}
    \item Reducing the algebra of distinctly nuanced algebraic objects onto simple
          arithmetic is the \textit{sine qua non} of the mathematics.
    \item The lofty ambition of all esoteric fields should be model-reduction to
          arithmetic.
    \item Whatever wanton discretion lead to the promulgation of arcane nomenclature and
          formalism must collapse to the imperative of simplicity.
    \item Nothing is simpler than the immeasurable simplicity of numbers and nothing is
          more understood.
    \item The goal of all mathematics should be the reduction of complicated objects to
          an algebra that is isomorphic to the natural numbers.
\end{enumerate}

And with that prelude I outline my method for reduction of iterated
polynomials.

\subsection{Polynomials as as Testing Ground}
As I have belabored to death at this point, my choice of polynomials as the
testing ground for my syntactic approach is not particularly stimulating. I
picked a well understood object to play with. Nothing beyond a superficial
inclination on my part can justify this choice. While my approach can be said
to yield tangible results in its application to polynomials, it is left as an
open question whether a just approach formulated for different class of objects
will yield similar productivity. I believe it will but have no desire to prove
it. That will take more quiet research on my part before I am ready to divulge
any success in other areas.

\subsection{Approach Proper}
Let the following definition be the starting point of our discussion.

\begin{equation}
    f(x)=ax^2 + bx + c, \quad f: \mathbb{R} \rightarrow \mathbb{R} \land a,b,c \in \mathbb{R}
\end{equation}

Supposed we wanted to find the closed form of $f(x)$ for $n$ iterations. That
would be quite an ordeal. While easily expressible symbolically, $f(x,n)$ where
$n \in \mathbb{N}$, the prospect of computing the closed form is of a degree
more onerous. For that reason, we will start simpler monomials to represent the
undertaking in a more accommodating context. Let the following three monomials
guide our discourse.

\begin{equation}
    f_1(x) = a_1, \quad f_1: \mathbb{R} \rightarrow \mathbb{R} \land a \in \mathbb{R}
\end{equation}

\begin{equation}
    f_2(x) = a_2 \cdot x, \quad f_2: \mathbb{R} \rightarrow \mathbb{R} \land a \in \mathbb{R}
\end{equation}

\begin{equation}
    f_3(x) = a_3 \cdot x^2, \quad f_3: \mathbb{R} \rightarrow \mathbb{R} \land a \in \mathbb{R}
\end{equation}

Monomials, by their simple structure, can easily yield a closed form through
modest effort. The following are the closed forms of the monomials.

\begin{align}
    f_1(x) = a_1, \quad f_1: \mathbb{R} \rightarrow \mathbb{R} \land a \in \mathbb{R} \notag \\
    f_1(x,n) = a_1
\end{align}

\begin{align}
    f_2(x) = a_2x, \quad f_2: \mathbb{R} \rightarrow \mathbb{R} \land a \in \mathbb{R} \notag \\
    f_2(x,n) = a_2 \cdot f_2(x,n-1) \notag                                                    \\
    f_2(x,n) = {a_2}^{n} \cdot x
\end{align}

\begin{align}
    f_3(x) = a_3 \cdot x^2, \quad f_3: \mathbb{R} \rightarrow \mathbb{R} \land a \in \mathbb{R} \notag \\
    f_3(x,n) = a_3 \cdot f_3(x,n-1) \notag                                                             \\
    f_3(x,n) = {a_3}^{2n - 1} \cdot x^{2n}
    \label{eq:closed-form-3}
\end{align}

Beauty is concise. And if polynomials behaved accordingly, closed forms would
be easier to compute. But alas, polynomials are not monomials. We can still
extract general tendencies by perusing the pattern of the closed forms. The
coefficients for example become parameters in the final form. They are fed
through the iterated function and come out with just as iterated as the
variable $x$ proper. This is so for amenable to our normal intuition and does
not deviated from any of our expectations. What polynomials have in store for
us however complicate the iterative story.

Scaffolding ontop of the simple monomial substrate is the vexing existence of
addition. I have much theory and philosophy to dissert on the nature of
addition, for example how addition is semantic equivalent to append by way of a
list; for now, we keep in mind that addition is not a number. Expressed
formally, addition is,

\begin{equation}
    +: \mathbb{P}^\alpha \times \mathbb{P}^\beta \rightarrow \mathbb{P}^\gamma, \quad \alpha, \beta, \gamma \in \mathbb{N} \land \gamma \in \{\alpha, \beta\}
\end{equation}

In this context, we are concerned with addition as a binary operation on
monomials that returns another monomial. The degree of the monomial is the
largest of the two operands.

Naturally, we would seek to find some isomorphism of addition to numbers. Set
more informally, we would like to treat addition as operated by iteration the
way numbers are operated by 1-arity functions. There is a rich field of theory
targeting this intuition. Operator theory seeks to understand the structure and
dynamics of functions when applied to an function. The operand function is
understood to be a member of some space of functions and operator function,
called an operator, is understood to be the mechanism by which the operand is
mapped to another space. That is all I will say about that now, suffice to add
there is beauty in this approach.

What I want to accomplish is even more basic than operators and function
spaces. I proclaim that there is a tendency of tracking the evolution of a
binary operation across iterations in the same way that there is a tacit
pattern for numbers. Let's think about this with solid examples. Let the
following be our starting points.

\begin{equation}
    n_1 \in \mathbb{N}
\end{equation}

\begin{equation}
    \lambda_1: \mathbb{N} \rightarrow \mathbb{N}
\end{equation}

Given a simple equation like $\lambda_1$, application of this function $n_1$
can be symbolically expressed with $\lambda_1(n_1)$. Applying the inverse would
like this: $\lambda_1^{-1}(\lambda_1(n_1))$. The inverse is the function that
when applied to the result of $\lambda_1$ yields $n_1$. The inverse is
$\lambda_1^{-1}$. The full syntax for applying the inverse is,

\begin{equation}
    (\lambda_1^{-1} \circ \lambda_1) \quad n_1
\end{equation}

Beautiful. This yields our original operand, $n_1$. Supposed we wanted to
analyze the type signature of the composed object. Nevermind that the
composition of an inverse with the original would annihilate itself to the
identity. Here it is.

\begin{equation}
    (\lambda_1^{-1} \circ \lambda_1): \mathbb{N} \rightarrow \mathbb{N} \rightarrow \mathbb{N}
    \label{eq:comp}
\end{equation}

I brought up the notion of type because by this point, its mention seems more
than proper. In building the type up from the ground, we must keep in mind that
the inverse takes a function that already has its own type. What \ref{eq:comp}
demonstrates is that we can easily feed a function another function and
retrieve a semantically coherent interpretation as a result. Perchance we can
provide some similar resolution to the problem of addition.

The first thing we would do is curry the function. This is a common practice in
functional programming to reduce a function of multiple arity to a series of
functions of arity 1. The following is the curried form of addition.

\begin{equation}
    +: \mathbb{P}^\alpha \rightarrow \mathbb{P}^\beta \rightarrow \mathbb{P}^\gamma, \quad \alpha, \beta, \gamma \in \mathbb{N} \land \gamma \in \{\alpha, \beta\}
\end{equation}

Again, let's take a moment to appreciate the beauty of this. There is an
intrinsic aesthetic quality to symbolically manipulating functions and objects
in general. However, we must part ways somewhat with this enthusiastic talk of
beauty. While we the symbols we have accumulated do indeed express the
definition properly, we are still far away from a concrete platform for
rendering iterated forms. Do not confuse symbolic semantic capture with
functional mechanism of achieving the computational result.

How can we leverage the curried form of addition to achieve our goal? Well, it
helps think of mathematical expressions as a sequence of operations performed
on some operand. Presented symbolically, we can express this as,

\begin{equation}
    \phi: \Gamma_\alpha \rightarrow \Gamma_\omega
\end{equation}

Think of $phi$ as a full-fledged program that a programmer would write up to
achieve some task. The set $\Gamma_\alpha$ captures the input to the program
and the set $\Gamma_\omega$ captures the output. Specifying the individual
steps as a sequence of operations, the program can be expressed as,

\begin{equation}
    \phi = \phi_1 \circ \phi_2 \circ \phi_3 \circ \dots \circ \phi_n
\end{equation}

The type signature of the program is as follows,

\begin{equation}
    \Gamma_\alpha = \Gamma_1
\end{equation}

\begin{equation}
    \Gamma_\omega = \Gamma_2
\end{equation}

\begin{equation}
    \phi_1: \Gamma_1 \rightarrow \Gamma_2
\end{equation}

\begin{equation}
    \phi_2: \Gamma_2 \rightarrow \Gamma_3
\end{equation}

\begin{equation}
    \phi_3: \Gamma_3 \rightarrow \Gamma_4
\end{equation}

\begin{equation}
    \phi_n: \Gamma_n \rightarrow \Gamma_{n+1}
\end{equation}

\begin{equation}
    \phi: \Gamma_1 \rightarrow \Gamma_2 \rightarrow \Gamma_3 \rightarrow \dots \rightarrow \Gamma_{n-2} \rightarrow \Gamma_{n-1} \rightarrow \Gamma_n
\end{equation}

\begin{equation}
    \phi: \Gamma_\alpha \rightarrow \Gamma_2 \rightarrow \Gamma_3 \rightarrow \dots \rightarrow \Gamma_{n-1} \rightarrow \Gamma_n \rightarrow \Gamma_\omega
\end{equation}

The composition of functions is associative and computation can begin at any
point and spread outwards.

Gorgeous. A function as a sequence of types is useful for breaking down an
operation to the metaphorical sum of its parts. We want to be able to do this
to polynomials. Not just in their structured operational form but in their
iterative dynamic one as well. Given \ref{eq:closed-form-3}, we can express the
iteration by following the type signature of the incremental iterations.
Starting from \ref{eq:closed-form-3},

\begin{equation}
    f_3(x) = a_3 \cdot x^2, \quad f_3: \mathbb{R} \rightarrow \mathbb{R} \land a \in \mathbb{R} \notag
\end{equation}

\begin{equation}
    f_3(x,n) = a_3 \cdot f_3(x,n-1) \notag
\end{equation}

\begin{equation}
    f_3(x,n) = {a_3}^{2n - 1} \cdot x^{2n} \notag
\end{equation}

\begin{equation}
    f_3(x,0) = x
\end{equation}

\begin{equation}
    f_3(x,1) = a_3 \cdot x^2
\end{equation}

\begin{equation}
    f_3(x,2) = a_3^3 \cdot x^4
\end{equation}

\begin{equation}
    f_3(x,3) = a_3^5 \cdot x^6
\end{equation}

Let's take do proper accounting of the degree of the polynomial across
iterations.

\begin{equation}
    f_3 \in \mathbb{P}^2
\end{equation}

\begin{equation}
    f_3(x,1) = f_3
\end{equation}

\begin{equation}
    f_3(x,2) \in \mathbb{P}^4
\end{equation}

\begin{equation}
    f_3(x,3) \in \mathbb{P}^6
\end{equation}

Expressed as a sequence of operations, the iteration of monomial becomes,

\begin{equation}
    f_3(x,n) \in \mathbb{P}^{2n}
\end{equation}

Let's define a helper type signature to aid in writing down the sequence type
signature.

\begin{equation}
    \Delta: \mathbb{T} \times \mathbb{N} \rightarrow \mathbb{T}^n
\end{equation}

$\Delta$ is a type function that accepts some type, from the set of types
$\mathbb{T}$, and a natural number. I will use this to produce the type of
degree of the polynomial.

\begin{equation}
    f_3(f_3,n): \Delta(\mathbb{P}, 2) \rightarrow \Delta(\mathbb{P}, 4) \rightarrow \dots \rightarrow  \Delta(\mathbb{P}, 2n-2) \rightarrow \Delta(\mathbb{P}, 2n)
\end{equation}

Note that the type signature of the polynomial holds even for polynomials.
While the example I used was for a simple $\mathbb{P}^2$ monomial, the same
type holds for any polynomial. This is because the degree of the polynomial is
purely a function of the number of iterations and the highest degree of the
polynomial. So it seems we made some progress. We can now express the type
signature for a polynomial as a sequence of operations.

Thus, given some arbitrary polynomial of degree $n$, we can express the type
signature as,

\begin{equation}
    f_n(f_n,m): \Delta(\mathbb{P}, n) \rightarrow \Delta(\mathbb{P}, n^m)
\end{equation}

Meditate on the symbolic elegance of such a statement. We have determined what
the type of an iterated polynomial is by pure symbolic expression of its own
degree and number of iterations. That is a beautiful thing to behold but we
still do not have a way to express the actual computation still.

Now, with all this build up, I think we are ready to share the secret. Finally,
now is the time to reveal the Copernican Revolution in Syntax. Much in the
spirit of G\"odel's numbering scheme, we can assign a number to each of our
operations that appear in a polynomial. The operations in a polynomial are
addition, multiplication, and exponent. Rather than symbolically numbering it
based on where the symbols for each operation appears, we will assign a scheme
ourselves. Quite frankly, we don't care about the operations syntactically. We
know they are contained somewhere in the expression. They key is not so much
the position but the relation of each of the operation to the one another.

The technique is to apply an ordering to the operations. The order should not
be confused with order of operations. While it is related, I care more about
reducing the overhead of representing and computing each operation. I would
like to reduce all three operations to pure symbolic function of computing a
more primitive composite. Let the following definition be our guiding light.

\begin{equation}
    \mathbb{O} = \{+, \cdot, ^\circ\}
\end{equation}

\begin{equation}
    \pi(0) = +
\end{equation}

\begin{equation}
    \pi(1) = \cdot
\end{equation}

\begin{equation}
    \pi(2) = ^\cdot
\end{equation}

\begin{equation}
    \pi(0,1) = +
\end{equation}

\begin{equation}
    \pi(0,2) = + \circ + = \pi(0) \circ \pi(0)
\end{equation}

\begin{equation}
    \pi(1,0) = +
\end{equation}

\begin{equation}
    \pi(1,1) = \cdot
\end{equation}

\begin{equation}
    \pi(1,2) = \cdot \circ + = \pi(1) \circ \pi(1)
\end{equation}

\begin{equation}
    \pi(1,n) = \pi_1(1) \circ \pi_2(1) \circ \dots \circ \pi_{n-1}(1) \circ \pi_n(1)
\end{equation}

This is all fine and good. I have sequenced the operations in a given order.
This particular should not require much justification. The "smaller" operation
of addition is at the bottom and the "larger" operation of exponentiation is
the biggest one. All I did is number them starting 0 and incrementing by 1. The
function $\pi$ is a function that returns a function. Think of it has a
convenient hash to store the operations in a given order. When we add $n$ as
second operand to $\pi$, we are adding the ability to iterate our operation.

Now here comes the real magic. Here it is.

\begin{equation}
    \pi(i) = \mathbb{O}_{i=0}^{n_i} \pi(i-1)
    \label{eq:pi-reduc}
\end{equation}

This equation is my contribution. Let me explain the vague terms and then I
will continue to explain the equation.

\begin{enumerate}
    \item $\pi(i)$ is a function that returns a function.
    \item $i$ is a natural number used to index $\pi$.
    \item $\mathbb{O}$ is my ad hoc way of composing some set of operations. In
          this example, I am using to compose a function with itself
    \item $n_i$ is the number of times to compose the function with itself. What
          $n_i$ is not important. It is a natural number that determines how many
          times to compose the function with itself. And the number defines how many
          iterations are needed to "reach" the next operation in the sequence.
    \item $\pi(i-1)$ is the function that is being composed with itself.
          Specifically, the previous function in the ordering is being iterated to
          yield the next operator in the sequence.
\end{enumerate}

While it is rough in some ways, I think I have achieved something significant.
Behold my gift to you. It is this equation and symbols.

The idea behind it is that any operations that we are currently studying can be
put in some ordering. The order can be any you desire but should primarily
reflect some intuitive notion of the "size" and ranking of the operation.
Ultimately, we are the arbiter of our symbols. We assign at our discretion and
compute at our leisure. A comfortable arrangement can similarly be made with
any class of algebraic objects. Now lets apply our little trick. To a simple
polynomial of degree 1. Let us play with the following definition.

\begin{equation}
    g(x) = 2 \cdot x + 3
\end{equation}

\begin{equation}
    g(x) = \pi(0)(\pi(1)(2,x), 3)
\end{equation}

\begin{equation}
    g(x,0) = x
\end{equation}

\begin{equation}
    g(x,1) = g(x) = \pi(0)(\pi(1)(2,x), 3)
\end{equation}

\begin{equation}
    g(x,2) = g(g(x)) = \pi(0)(\pi(1)(2,x), 3) \circ \pi(0)(\pi(1)(2,x), 3)
    \label{eq:iter2}
\end{equation}

Note that the general difficulty of this problem, the problem of finding a
closed-form for iterating polynomials, comes from the onerous requirement of
having to treat a polynomial at one point as a noun and operand and
simultaneously as a verb and operator. This is the crux of the problem.

We need to make another adjustment to be able to overcome a hurdle. You can see
in \ref{eq:iter2} we almost have a streamlined account of pure function
composition. If it weren't those pesky numbers, we would have a pure sequence.
Well, for the context of what we are trying to accomplish, we don't care what
the particular non-function operand is. All we care is the particular ordering
of the sequence. So let us make a small adjustment to our definition. Not only
are we going to curry our functions again, we will abstract away wherever we
see a number. I justification for this is that numbers are terminal nodes in
our computation tree. We follow the thread of execution until we reach a
number; otherwise, we thread into another function to compute the subtree. So
the new definition is,

\begin{equation}
    g(x,2) = (g \circ g) x = ((\pi(0) \circ \pi(1)) \circ (\pi(0) \circ \pi(1))) x
    \label{eq:iter3}
\end{equation}

That is much simpler. We can see the general order of operations as they are
applied in the final expression. The expression when computed should yield what
the original polynomial would yield, iterated once. Closer and closer do the
symbols encroach on a more elegant basis for computation on pure function
composition alone.

What is currently thwarting our efforts for simplifying the symbols of
composition is the gap that exists between the terms $\pi(0)$ and $\pi(1)$. If
we could combine the two terms amongst themselves, we could achieve a modicum
of simplification that may improve our chances of computing the final
expression. However, we need to know what the dynamics is when we have the
composition expression $\pi(0) \circ \pi(1)$. We need to know what the
composition of two separate functions is. Can they be reduced? Or better yet,
can they be reordered without affecting the final result? In other words, does
the algebra for composition allow for reordering? Given that we have
constructed and manipulated our symbols at our pleasure, perhaps we can
discover some rule that would permit judicious reordering so that like terms
can combine. Here is the possible algebraic rule that provides a equitable
denouement to our dilemma.

\begin{equation}
    \pi(0) \circ \pi(1) = \pi(0) \circ \mathbb{O}_{i=0}^{n_i} \pi(0)
\end{equation}

\begin{equation}
    ((\pi(0) \circ \pi(1)) \circ (\pi(0) \circ \pi(1)))  = \pi(0) \circ \mathbb{O}_{i=0}^{n_i} \pi(0) \circ \pi(0) \circ \mathbb{O}_{i=0}^{n_i} \pi(0)
\end{equation}

\begin{equation}
    m = 2 \cdot n_i + 2
\end{equation}

\begin{equation}
    \pi_1(0) \circ \pi_2(0) \circ \dots \circ \pi_{m-1}(0) \circ \pi_m(0)
\end{equation}

Using subscript to properly index the function, they can be safely ignored.
What we have done is the final composition of a polynomial with multiplication
and addition only. Even though we have only referenced two different types of
functions, the technique we used to reduce a function of $\pi(1)$ to $\pi(0)$
can be applied to any function $\pi(i), \quad i \in \mathbb{N} \land i > 0$.
Given a function $\pi(i)$, we can reduce it to $\pi(i-1)$ by \ref{eq:pi-reduc}.
Any function can be reduced to a function of $\pi(0)$.

That is quite a quaint development but I rather not have to convert all
functions to $\pi(0)$ before I can compute the final expression. Much to
unwieldy and for a algebraic technique, it leaves a lot to be desired. Instead,
lets focus on what concession we can make to extract a proper reordering to
move around immiscible terms.

Why should we want this? Well, naively our intuition is guiding us towards
combining like terms so that the final operation is is sequenced by function.
First we compute all the functions of $\pi(1)$, then all the functions of
$\pi(0)$, and so on. Given our propensity to prefer an ordering that coincides
with our intuition, it is rather fortunate that the gods of algebra posit a
sort of rank ordering for execution. We compute the "larger" $\pi(i)$ first and
then the "smaller" $\pi(i-1)$, and so on. If we had the larger $\pi(2)$,
corresponding to exponentiation, we would strive as result of manipulating our
algebra to compute it and then passing the result to the smaller $\pi(1)$.
Naturally this is what we get.

Let's say we had the following polynomial,

\begin{equation}
    f(x) = 2 \cdot x^2 + 2 \cdot x + 1
\end{equation}

The previous paragraph should narrowed down our perspective to polynomials as
computation. More Specifically, we can think of polynomial as rank-ordered
computation of $\pi(i) \quad i \in {0,1,2} $, and so on. The rank ordering This
corresponds to computing all $\pi(2)$ regardless of where they appear in the
expression. Of course, given that we are working with polynomials, we know
quite clearly where all the $\pi(2)$ are. But the point of this exercise is
that their order in situ the expression does not matter. It is their rank
ordering that matters. While I lack further justification for why this should
be the case, there is something seductive about executing the "larger".
functions first.

Before we continue, let me justify the rank order computation. A rich field of
study is the generalization of the function. While we can indeed maintain that
a function $z(x), \quad x \in \mathbb{R}$ is a function of a single real
variable, we not limited by our own pronouncement. What we decide is
exclusively left to our own prerogative and what we can do is to our
imagination. One can imagine the operand fed to function as some unprocessed,
abstract, amorphous object. All that is needed for $z(x)$ to be valid for an
algebraic object is that the operational semantics. And since we are the
arbiters of all things definitions, we can easily construct a valid
interpretations for our own purposes. Now assuming that $z(x)$ is a polynomial,
we can create a valid interpretation for objects other than numbers. As
mentioned before, there already is a field of study that has capitalized on
this opportunity. Some canonical objects that have studied are matrices,
vectors, tensors, functions, operators and so on. The amorphous argument is
then fed to conveyor belt of operations. While this is a deviation from our
study, a satisfying symbolic sentence that more or less captures the essence of
this intuition is the following.

\begin{equation}
    z(x) = \alpha_0 \cdot \pi(0) + \alpha_1 \cdot \pi(1) + \alpha_2 \cdot \pi(2)
\end{equation}

In a perfect world, all operators would be commutative and the forthwith
declarations of involved operators would be followed with a simple linear
algebraic expression that impresses the vector space of computation. Alas, we
do not reside in the perfect paradise quite yet.

Back to our polynomial operator. Ignoring the specific number of invocations
for each of the $\pi(i)$ functions, we can write the following expression to
specify the ordering of the computation.

\begin{equation}
    f(x) = \pi(0) \circ \pi(1) \circ \pi(2)
\end{equation}

Expressed using our composition summation notation, we can write the following

\begin{equation}
    f(x) = \mathbb{O}_{i=2}^{0} \pi(i)
\end{equation}

\section{Suggested Points of Departure for Future Research}
\begin{enumerate}
    \item Is this a viable approach to the problem of iterated functions?
    \item Is there any counter example which demonstrates that this approach does not
          work for polynomials?
    \item Is there a way to generalize this approach to other types of functions?
    \item Is there a way to generalize this approach to other types of algebraic objects?
\end{enumerate}