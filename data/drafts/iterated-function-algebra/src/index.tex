\maketitle
\pagebreak

\begin{abstract}
    One of the greatest intellectual crimes to beset us in the 20th-century was
    the premature death of the formalist program. The millennia old dream of
    \textit{solving math} was never realized as our efforts fell short of our
    ambition. David Hilbert, with all his grace, his effulgent brilliance, his
    professional magnanimity, and his unflinching dedication, was left with only
    disgruntled disappointment. The formalist program was a noble effort, held
    aloft by the unyielding conviction and charisma of the foremost
    mathematicians of the age. This forlorn vignette is rendered at least
    somewhat emotionally digestible by the developments that followed. The dream
    of ''syntax is all'' became partially realized by the contributions of Haskell
    Curry, Alonzo Church, Stephen Kleene, Moses Schönfinkel and others. With
    their syntactic approach to mathematics, we capture the compositional beauty
    of infinity in our humble symbol. And thus, we compile the syntactic face of God.
\end{abstract}
\pagebreak
\tableofcontents
\pagebreak

\section{Formalism}
\subsection{The Dream}
\subsection{The Nightmare}
\subsection{The Gentle Promenade}

\section{The Dead Shall Rise Again}

\section{Introduction}
Many people take for granted the fact that 2+2 equals 4. But how do we know
this is true? In this paper, we will provide a rigorous proof of this fact
using only the basic axioms of arithmetic.

\section{Points}
\begin{enumerate}
    \item Reducing the algebra of distinctly nuanced algebraic objects onto simple
          arithmetic is the \textit{sine qua non} of the mathematics.
    \item The lofty ambition of all esoteric fields should be model-reduction to
          arithmetic.
    \item Whatever wanton discretion lead to the promulgation of arcane nomenclature must
          collapse to the imperative of simplicity.
    \item Nothing is simpler than the immeasurable simplicity of numbers.
\end{enumerate}

\section{Proof}
First, we start with the axioms of arithmetic, which include the following:
\begin{enumerate}
    \item The commutative property of addition: $a+b=b+a$ for any real numbers $a$ and
          $b$.
    \item The associative property of addition: $(a+b)+c=a+(b+c)$ for any real numbers
          $a$, $b$, and $c$.
    \item The identity property of addition: $a+0=a$ for any real number $a$.
    \item The existence of additive inverses: for any real number $a$, there exists a
          real number $-a$ such that $a+(-a)=0$.
\end{enumerate}

Using these axioms, we can prove that 2+2=4 as follows:
\begin{align*}
    2+2 & = (1+1)+(1+1) & \text{(by definition of 2)} \\
        & = 1+(1+1)+1   & \text{(by associativity)}   \\
        & = 1+1+(1+1)   & \text{(by associativity)}   \\
        & = (1+1)+2     & \text{(by associativity)}   \\
        & = 4           & \text{(by definition of 4)}
\end{align*}

Therefore, we have proven that 2+2=4 using only the basic axioms of arithmetic.

\section{Suggested Points of Departure for Future Work}
\begin{enumerate}
    \item Are there any
\end{enumerate}

\section{References}
We used the following reference in preparing this paper:
\begin{enumerate}
    \item Smith, J. (2001). Basic Arithmetic Axioms. Journal of Mathematics, 3(2), 47-51.
\end{enumerate}

\section{Acknowledgments}

The author would like to thank the mathematical community for their continuous
The author would like to thank the mathematical community for their continuous
efforts to advance the understanding of fundamental mathematical concepts.
\begin{thebibliography}{9}

    \bibitem{Peano1889}
    Giuseppe Peano.
    \textit{Arithmetices Principia, Nova Methodo Exposita}.
    Bocca, 1889.

\end{thebibliography}