\maketitle
\pagebreak

\begin{abstract}
    One of the greatest intellectual crimes to beset us in the 20th-century was
    the premature death of the formalist program. The millennia old dream of
    \textit{solving math} was never realized as our efforts fell short of our
    ambition. David Hilbert, with all his grace, his effulgent brilliance, his
    professional magnanimity, and his unflinching dedication, was left with only
    disgruntled disappointment. The formalist program was a noble effort, held
    aloft by the unyielding conviction and charisma of the foremost
    mathematicians of the age. This forlorn vignette is rendered at least
    somewhat emotionally digestible by the developments that followed. The dream
    of \textbf{syntax is all} became partially realized by the contributions of Haskell
    Curry, Alonzo Church, Stephen Kleene, Moses Schönfinkel and others. With
    their syntactic approach to mathematics, we capture the compositional beauty
    of infinity in our humble symbol. And thus, we compile the syntactic face of God.
\end{abstract}
\pagebreak
\tableofcontents
\pagebreak

\section{The Tragedy of Formalism}
The death of the formalism program was a necessary tragedy for the advancement
of mathematics. Gone were the ensconced notions were the naive intuitions of
mathematical computation. Rather, it was the lack of such a computational
foundation for mathematics that needed to be addressed. It was around this time
that mathematicians began to rigorously define what it means for a statement to
be provable. Without a computational foundation, the notion of provability
rested on a tentative and ambiguous folklore. For this heroic death, we
inherited a proliferation of computable logical systems that could be used to
encode mathematical ideas and theorems. The program's death died for our crass
naivite and ensured the salvation of mathematics for the modern age. It is here
where we demark the beginning of computation.

\subsection{Poetry of Computation}
The poetry of computation is the poetry of the formalist program reborn.
Whether it be lambda calculus or combinatory logic, the accentuation of syntax
is what deliberately marks this change in tradition. What Frege set in motion
is that the story of math and logic is in part the story of syntax. And it is
with syntax that we seek to capture the semantics of mathematics. Whether we
behold the history of mathematical symbols establishing what we associate with
the mathematical arts, or modern compilers that bleed out the captured
semantics of a language to a hardware equivalent, syntax is the poetry of
computation. What the formalist program lacked in ultimate success, it made up
for in intuition and beauty.

\subsection{A Nightmare Embraced}
Of course, the program is not without its thorns. To assume that the program
must be a feasible one is to posit that mathematics is devoid of content beyond
its superficial and meaningless syntax. The inevitable progression of this
species of thought is that mathematics is a mere syntactic game. Any semantic
model we attach to this game must submit to the reality of being mere fiction.
And if this holds, then our notions of mathematics fall into the void of
nihilism and thus embrace the nightmare of extinction. This nightmare becomes a
wonderful dream only when coupled with a more human focused approach to
mathematics, such as intuitionism. I won't elaborate on what this latter
program entails suffice to say that it dulls the sting of the void. Once we
compliment our syntax program with a phenomenological narcotic, we can
commandeer the development of mathematics where the syntax demands and inject
our semantic creature comforts as an afterthought. After all, mathematics is a
inductive, deductive and analytical art; there is no room for a bleeding heart.

\subsection{A Gentle Promenade Down The Road to Perdition}
Let us follow this dreamy reverie down the whirlpool of damnation. Retaining
the insight that our formalist program may not condemn us to the void by a
serendipitous pairing to some phenomenological strand of thought. I am not
concerned what flavor or brand of special pleading is provisioned. Consider the
program plead. Now, envision if you might that the program were not doomed to
the ash bin of history. In this alternative hypothetical universe, the program
succeeds. We have successfully reduced all of mathematics up to that point to
pure symbolic function of syntax. Semantics thus submits readily to capture by
the mystical symbol. The program is a success. An element of this universe is
in substance some ready made mathematical object that computes algebra and
reduces it to theorems of our liking. Our appetite for conclusive proofs and
pure deductive reasoning is sated. That is the dream. Well, if I have not
belabored enough (forgive me!), I modestly posit that claims of the death of
this program are of mild exaggeration. The program is not dead, it is merely
gone dormant while our syntactic tools sharpened.

\section{Building A Sea-fairing Vessel out of a Corpse}
Dear reader, may you forgive me for my crass verbosity. I fear it is the only
way my sinister little thoughts can find their way to a wider audience.
Consider it a concession well spent. I do indeed hold that the purview of the
formalist institution continues full steam, unencumbered by the romance of
larger narratives. Again, I outlined in cursory detail the different syntactic
program that explored what is possible with theory. As maligned as syntax is as
a foundation for mathematics, it has not heeded the call to abolish itself out
of existence. Progress, once tasted, is not easily relinquished. What has been
thwarted by historical account conclusively is the naive version of the
program. By necessity, any research that follows in the aftermath must deviate
in substance from the initial thrust of the program. The impetus of critique
imparts a degree of resiliency to subsequent efforts. And it is these efforts
that carry the banner of formalism, notwithstanding the more judicious
features.

\subsection{The Naive Intuition of the Initial Program}
Why did the program fail? Did it succumb to the bleeding that sparring with
intuitionism? Probably not as both programs are rather dead in their initial
manifestation. I hold the notion that the intuition behind each of these
efforts complimented each other pleasantly and should not have been competitors
for our hearts and minds. But such is life when a post-mortem autopsy yields
two corpses instead one. It is an incident of history I feel that David Hilbert
triumphed and that he dueled the other school in the first place

\subsection{Rescinding The Full-throated Ambition of Formalism}
\subsection{A Mild Tangent on the Intuitionism Compliment}

\section{Ambitions Laid Bare}
\subsection{Outline of my Ambitions}
\begin{enumerate}
    \item Reducing the algebra of distinctly nuanced algebraic objects onto simple
          arithmetic is the \textit{sine qua non} of the mathematics.
    \item The lofty ambition of all esoteric fields should be model-reduction to
          arithmetic.
    \item Whatever wanton discretion lead to The promulgation of arcane nomenclature must
          collapse to the imperative of simplicity.
    \item Nothing is simpler than the immeasurable simplicity of numbers.
\end{enumerate}

\subsection{Polynomials as as Testing Ground}
\subsection{}

\section{Suggested Points of Departure for Future Research}
\begin{enumerate}
    \item Are there any
\end{enumerate}

\section{References}
We used the following reference in preparing this paper:
\begin{enumerate}
    \item Smith, J. (2001). Basic Arithmetic Axioms. Journal of Mathematics, 3(2), 47-51.
\end{enumerate}