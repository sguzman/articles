\maketitle
\pagebreak

\begin{abstract}
    One of the greatest intellectual crimes to beset us in the 20th-century was
    the premature death of the formalist program. The millennia old dream of
    \textit{solving math} was never realized as our efforts fell short of our
    ambition. David Hilbert, with all his grace, his effulgent brilliance, his
    professional magnanimity, and his unflinching dedication, was left with only
    disgruntled disappointment. The formalist program was a noble effort, held
    aloft by the unyielding conviction and charisma of the foremost
    mathematicians of the age. This forlorn vignette is rendered at least
    somewhat emotionally digestible by the developments that followed. The dream
    of \textbf{syntax is all} became partially realized by the contributions of Haskell
    Curry, Alonzo Church, Stephen Kleene, Moses Schönfinkel and others. With
    their syntactic approach to mathematics, we capture the compositional beauty
    of infinity in our humble symbol. And thus, we compile the syntactic face of God.
\end{abstract}
\pagebreak
\tableofcontents
\pagebreak

\section{The Tragedy of Formalism}
The death of the formalism program was a necessary tragedy for the advancement
of mathematics. Gone were the ensconced notions were the naive intuitions of
mathematical computation. Rather, it was the lack of such a computational
foundation for mathematics that needed to be addressed. It was around this time
that mathematicians began to rigorously define what it means for a statement to
be provable. Without a computational foundation, the notion of provability
rested on a tentative and ambiguous folklore. For this heroic death, we
inherited a proliferation of computable logical systems that could be used to
encode mathematical ideas and theorems. The program's death died for our crass
naivite and ensured the salvation of mathematics for the modern age. It is here
where we demark the beginning of computation.

\subsection{Poetry of Computation}
The poetry of computation is the poetry of the formalist program reborn.
Whether it be lambda calculus or combinatory logic, the accentuation of syntax
is what deliberately marks this change in tradition. What Frege set in motion
is that the story of math and logic is in part the story of syntax. And it is
with syntax that we seek to capture the semantics of mathematics. Whether we
behold the history of mathematical symbols establishing what we associate with
the mathematical arts, or modern compilers that bleed out the captured
semantics of a language to a hardware equivalent, syntax is the poetry of
computation. What the formalist program lacked in ultimate success, it made up
for in intuition and beauty.

\subsection{A Nightmare Embraced}
Of course, the program is not without its thorns. To assume that the program
must be a feasible one is to posit that mathematics is devoid of content beyond
its superficial and meaningless syntax. The inevitable progression of this
species of thought is that mathematics is a mere syntactic game. Any semantic
model we attach to this game must submit to the reality of being mere fiction.
And if this holds, then our notions of mathematics fall into the void of
nihilism and thus embrace the nightmare of extinction. This nightmare becomes a
wonderful dream only when coupled with a more human focused approach to
mathematics, such as intuitionism. I won't elaborate on what this latter
program entails suffice to say that it dulls the sting of the void. Once we
compliment our syntax program with a phenomenological narcotic, we can
commandeer the development of mathematics where the syntax demands and inject
our semantic creature comforts as an afterthought. After all, mathematics is a
inductive, deductive and analytical art; there is no room for a bleeding heart.

\subsection{A Gentle Promenade Down The Road to Perdition}
Let us follow this dreamy reverie down the whirlpool of damnation. Retaining
the insight that our formalist program may not condemn us to the void by a
serendipitous pairing to some phenomenological strand of thought. I am not
concerned what flavor or brand of special pleading is provisioned. Consider the
program plead. Now, envision if you might that the program were not doomed to
the ash bin of history. In this alternative hypothetical universe, the program
succeeds. We have successfully reduced all of mathematics up to that point to
pure symbolic function of syntax. Semantics thus submits readily to capture by
the mystical symbol. The program is a success. An element of this universe is
in substance some ready made mathematical object that computes algebra and
reduces it to theorems of our liking. Our appetite for conclusive proofs and
pure deductive reasoning is sated. That is the dream. Well, if I have not
belabored enough (forgive me!), I modestly posit that claims of the death of
this program are of mild exaggeration. The program is not dead, it is merely
gone dormant while our syntactic tools sharpened.

\section{Building A Sea-fairing Vessel out of a Corpse}
Dear reader, may you forgive me for my crass verbosity. I fear it is the only
way my sinister little thoughts can find their way to a wider audience.
Consider it a concession well spent. I do indeed hold that the purview of the
formalist institution continues full steam, unencumbered by the romance of
larger narratives. Again, I outlined in cursory detail the different syntactic
program that explored what is possible with theory. As maligned as syntax is as
a foundation for mathematics, it has not heeded the call to abolish itself out
of existence. Progress, once tasted, is not easily relinquished. What has been
thwarted by historical account conclusively is the naive version of the
program. By necessity, any research that follows in the aftermath must deviate
in substance from the initial thrust of the program. The impetus of critique
imparts a degree of resiliency to subsequent efforts. And it is these efforts
that carry the banner of formalism, notwithstanding the more judicious
features.

\subsection{The Naive Intuition of the Initial Program}
Why did the program fail? Did it succumb to the bleeding that sparring with
intuitionism? Probably not as both programs are rather dead in their initial
manifestation. I hold the notion that the intuition behind each of these
efforts complimented each other pleasantly and should not have been competitors
for our hearts and minds. But such is life when a post-mortem autopsy yields
two corpses instead one. It is an incident of history I feel that David Hilbert
triumphed and that he dueled the other school in the first place. The syntactic
approach of Hilbert was a more pragmatic one. It focused on the material
conditions in which mathematics was practiced and developed. Syntax was the
conduit for his program because syntax was what conveyed the semantic
information of mathematics. To Hilbert, analysis of the syntax, and thus the
grammar of mathematics, was the key to understanding the exposition of a
computable basis. It is not to say that Hilbert believed mathematics to be the
mindless churning of symbols and syntax was the heart of mathematics. But in
the pliant concession of this profile of the art do we extract the possible
essence of a feasible program. In the rotten carcass of its demise, do we find
brilliant gems of insight that can be salvaged for future use.

\subsection{Rescinding The Full-throated Ambition of Formalism}
Doomed ambition need not be a death sentence and even carrion can provide a
cozy abode if for maggots alone. What precluded the longevity of the enterprise
is the pallid condition of research into syntax and formal languages.
Contemporaries would have to wait several decades until the illustrious formal
studies of grammars and languages produced more solid foundations. Hilbert
himself must have known this to some degree. His questions had a rather
informal air to them and left as an open question whether new mathematical
machinery would be needed to satisfy his demands. Given how much research was
put into syntax, it would be a daunting task if he knew how much literature it
would spawn. To wit: lambda calculus, combinatory logic, type theory, category
theory and so on. These are a few syntax-related or algebraic-focused
approaches to studying mathematics. The list is not exhaustive and the future
is sure to spawn more. Thus, it is here that we find our departure of his
program, respectfully pay homage to this effort, offer eulogies and then move
on to greener pastures.

\subsection{Seduction of Intuitionism}
While ultimately vindicated by the promulgation of later research, intuitionism
was dealt quite a fatal blow. Hilbert's words spoken against intuitionism
proved potent vigour against it. His champion of "more results", not less, was
the rhetoric of the day. I hope in our age, we see the flaws in such reasoning.
Intuitionism hoped to bring mathematics to us and create a more constructive
substrate to leverage for the communicating of mathematical reasoning. I admit
to prejudice hearing it from Hilbert himself and saw myself as opposed out of
principal such aspirational goals. But merit of extinguishing the practice of
mathematicians, as practiced for millennia, was at the time not compelling.
Strange it is today, to this humble observer, that the argument of ipso facto
posteriority would rule the day in discussion of advancing technology. Do we
say today that new programming languages should not be fashioned to not
eclipsed the interest of previous languages and programmers? Outrages! But such
was the time and such was curling of the monkey paw dealing intuitionism a
dishonorable discharge from the ranks of mathematics.

\section{Ambitions Laid Bare}
So what is it that I seek? Thus far we have arrived at two dead ends and no
light to elucidate our dark path. Have we fully exhausted our resolve? Of
course not. What we are now left is the raison d'etre of this document. With
everything I have said in heart and also with the motivations contained in the
initial undertakings, I now present my own ambitions for the future of
mathematics. I want a syntactic basis for mathematics. I want this to reduce
the development of this field as a matter of logical consequence extrapolated
from the choice of syntax. I also believe wholeheartedly the intuitionism
utopia of a constructive and human-digestible vantage point for tackling
analysis. It is in the spiritual synthesis in these things that some modicum of
genius is found. But I will leave that enterprise for some future date. Suffice
for now is my own modest ambitions.

\subsection{Princess in the Tower of Algebra}
It is my strong conviction that algebra as it is understood in mathematics is
substitute for other concepts. Algebra as a principal captures structure,
symmetry and dynamics of a system of objects. It is readily amenable to
analysis by syntax since formalism is indeed syntax itself. It is very tempting
to transgress the boundary of this observation and claim that algebra is only
syntax. I will leave that transgression for some other day. For now, let's
assume that algebra and algebras are pliant things and syntax is no censurious
approach to understanding them. What I am wrestling with, dear reader, is
generalizing the notion of algebra to a more general notion of structure
permitting computation. We have an algebra of our beloved objects and can with
exceeding ease compute properties with these systems. What if we had a algebra
for everything? What if we had an algebra for generating other algebras? That
licentious thought must sadly be put to rest for now. But I hope the temptation
of such a beauty will stay with you and haunt your languid summer days.

\subsection{Outline of my Ambitions}
My train of thought lead me to extrapolate from syntactic principals to more
general laws governing the dynamics of algebraic objects. I am effervescent
with excitement to share what I have found! My personal research had found
purchase in implementing syntactic methods for polynomials. Predilection proper
of mine lead me to pursue a novel approach to computing the closed form of
iterated polynomials. Knowing the particular polynomial function from the start
yielding the closed form. In outline and part, here follows my general line of
reasoning.

\begin{enumerate}
    \item Reducing the algebra of distinctly nuanced algebraic objects onto simple
          arithmetic is the \textit{sine qua non} of the mathematics.
    \item The lofty ambition of all esoteric fields should be model-reduction to
          arithmetic.
    \item Whatever wanton discretion lead to the promulgation of arcane nomenclature and
          formalism must collapse to the imperative of simplicity.
    \item Nothing is simpler than the immeasurable simplicity of numbers and nothing is
          more understood.
    \item The goal of all mathematics should be the reduction of complicated objects to
          an algebra that is isomorphic to the natural numbers.
\end{enumerate}

And with that prelude I outline my method for reduction of iterated
polynomials.
\subsection{Polynomials as as Testing Ground}
\subsection{}

\section{Suggested Points of Departure for Future Research}
\begin{enumerate}
    \item Are there any
\end{enumerate}