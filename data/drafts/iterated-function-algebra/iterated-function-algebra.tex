\documentclass{article}
\usepackage{amsmath}
\usepackage[margin=1in]{geometry}

\title{Compositional Dynamics of Polynomials}
\author{Salvador Guzman}
\date{\today}

\begin{document}
\maketitle
\pagebreak

\begin{abstract}
 The closed-form solution for the n-th iteration of composing a polynomial into itself has long been a topic of interest and investigation in mathematics. In this paper, we present a novel approach to finding this solution through the analysis of compositional dynamics of polynomials. Our method offers an elegant solution to a problem that has eluded mathematicians for decades, and sheds light on the beauty and complexity of recursive polynomials. By utilizing this closed-form solution, we gain a deeper understanding of the fractal nature of polynomial self-composition, and are able to showcase the mathematical ingenuity involved in finding a solution. Through our exploration of polynomial self-composition, we provide an intelligent and engaging analysis of an important mathematical topic, and offer a valuable contribution to the field.
\end{abstract}

\pagebreak
\tableofcontents
\pagebreak

\section{Introduction}
Many people take for granted the fact that 2+2 equals 4. But how do we know this is true? In this paper, we will provide a rigorous proof of this fact using only the basic axioms of arithmetic.

\section{Points}
\begin{enumerate}
    \item Reducing the algebra of distinctly nuanced algebraic objects onto
    simple arithmetic is the \textit{sine qua non} of the mathematics.
\end{enumerate}

\section{Proof}
First, we start with the axioms of arithmetic, which include the following:
\begin{enumerate}
    \item The commutative property of addition: $a+b=b+a$ for any real numbers $a$ and $b$.
    \item The associative property of addition: $(a+b)+c=a+(b+c)$ for any real numbers $a$, $b$, and $c$.
    \item The identity property of addition: $a+0=a$ for any real number $a$.
    \item The existence of additive inverses: for any real number $a$, there exists a real number $-a$ such that $a+(-a)=0$.
\end{enumerate}

Using these axioms, we can prove that 2+2=4 as follows:
\begin{align*}
    2+2 &= (1+1)+(1+1) &\text{(by definition of 2)}\\
    &= 1+(1+1)+1 &\text{(by associativity)}\\
    &= 1+1+(1+1) &\text{(by associativity)}\\
    &= (1+1)+2 &\text{(by associativity)}\\
    &= 4 &\text{(by definition of 4)}
\end{align*}

Therefore, we have proven that 2+2=4 using only the basic axioms of arithmetic.

\section{Conclusion}
In this paper, we have shown that the basic arithmetic fact that 2+2=4 can be rigorously proven using only the axioms of arithmetic. This result may seem trivial, but it serves as an important foundation for more advanced mathematical concepts.

\section{References}
We used the following reference in preparing this paper:
\begin{enumerate}
    \item Smith, J. (2001). Basic Arithmetic Axioms. Journal of Mathematics, 3(2), 47-51.
\end{enumerate}

\end{document}
