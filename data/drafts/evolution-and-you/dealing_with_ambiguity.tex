\documentclass[UTF8]{article}
\title{%
	Ambiguity and Its Discontents\\
	\large Dealing with the Manifold Vagaries of Ambiguity in Math
}
\author{Salvador Guzman Jr}
\date{\today}

\begin{document}
	\maketitle
	\tableofcontents
	
	\paragraph{Ambiguity}
	The ascent of technology in our times captures the perfect warrant to examine the pillar of this entire enterprise; mathematics. Far from being an arcane 
	
	\section{Whence Cometh Ambiguity}
		\subsection{Ambiguity is everywhere}
			\subsubsection{Ambiguity intolerance}
			\subsubsection{Ambiguity the tyrant}
			\subsubsection{Ambiguity inviting a response}
			\subsubsection{Our response to it}
	\section{Addressal}
		\subsection{Theory}
			\subsubsection{What is Theory}
			\subsubsection{What is the intention of theory}
			\subsubsection{Does it succeed}
		\subsubsection{Rigor}
			\subsubsection{What is Rigor}
			\subsubsection{Rigor as implementing an ambiguity-reduction process}

	
	\section{Typing Text}
	Since \LaTeX\ is a markup language, any text we
	type appears on the page, unless it contains
	one of the nine reserved characters of \LaTeX, listed
	below.
	\begin{verbatim}
		\ { } & _ ^ % ~ #
	\end{verbatim}
	If we want those characters to appear on the page, we
	must precede them by a backslash, or, in the special
	case of the backslash itself, we must use the 
	command \verb(\verb(.  In math mode, we can use the command
	\verb(\backslash(.
	
	Note that there are three kinds of dash-objects.  Hyphens 
	are very short, and are typed the way you would expect, using
	"-".  Dashes, as in the range 1--2, are wider, and are typed
	using \verb+--+.  If you feel a need for a wide---dash, you can
	use \verb+---+.
	
	\section{Units}
	Lengths in \LaTeX\ can be given in a number of Units.\\
	\begin{tabular}{ll}
		cm & Centimeters\\
		em & The width of the letter M in the current font\\
		ex & The height of the letter x in the current font\\
		in & Inches\\
		pc & Picas (1pc = 12pt)\\
		pt & Points (1in = 72pt)\\
		mm & Millimeters
	\end{tabular}
	
	\section{Space}
	The most direct way to make spaces is simply to use the
	\verb(\hspace( and \verb(\vspace( commands, for horizontal
	and vertical space, respectively.  Each takes one
	argument: a distance specification for the size of the space.
	The width or height of the space can be positive {\em or}
	negative.  Note that \verb(\vspace( can only be used in 
	vertical mode; that is, when you are starting a new paragraph
	or starting a float or doing something else that shifts text
	vertically.  It will not work in a line.
	
	\LaTeX\ also has a number of predefined spaces.  To produce a
	space with fixed width, and which cannot be used as a line break,
	you may use the \verb(~( character.  This would typically be used
	to separate initials of an author, or in other situations where
	we don't want to have a single letter or initial ending a line.
	More often, we don't mind a line break, and want the space to shrink
	or grow according to the justification requirements on the line.
	In that case we make a standard space using \verb(\ (; backslash-space.
	
	There are wider stretchy spaces available to us: \verb(\quad( and 
	\verb(\qquad(.  There is also a ``thin space'': \verb(\,(.
	The following words are separated by a thin space, a standard space,
	a quad, and a qquad, respectively.
	
	\begin{center}
		space\,space\ space\quad{space}\qquad{space}
	\end{center}
	
	There are also predefined vertical spaces: \verb(\smallskip(
	\verb(\medskip(, and \verb(\bigskip(, that behave as their
	names imply.
	
	There are also some exceptionally stretchy spaces that we can 
	use to push text around.  For example, the following line is 
	set using the command \verb(text\hfill text(.  The line below it 
	was set using \verb(text\hfill text\hfill text(.\\
	{text\hfill text}\\
	text\hfill text\hfill text\\
	You get the idea: \verb(\hfill( makes enough space to fill the line
	in question completely.  If more than one \verb(\hfill( appears
	on a line, then the two negotiate over how much space they each get.
	
	\section{Lines and Boxes}
	There are a variety of ways to make lines and boxes in \LaTeX.
	The most basic is to make a horizontal rule using \verb(\hrule(.
	\hrule
	\verb(\hrule( makes a new line, and fills it up with a horizontal line.
	If you don't want an entire line, you can use \verb(\hrulefill(, as
	in \hrulefill.\\
	This command works like \verb(\hfill(, but instead of filling with
	space, it uses a horizontal rule to fill the line.
	
	To make a box around some content, you can use the \verb(\framebox(
	command.  The framebox command puts a box around its argument,
	so that \verb(\framebox{text}( looks like \framebox{text}. It takes
	optional width and position arguments, so that
	\verb(\framebox[3in][l]{text}(
	appears as \\
	\framebox[3in][l]{text}.
	
	\section{Margins}
	This section is mistitled.  \LaTeX\ does not really do margins,
	so much as it places text.  It uses several variables in placing the
	text, which we can set.  For example, to make the left margin on
	all even-numbered pages 0.5 inches wider than the default (which is 
	1 inch), we would define\\
	\verb(\evensidemargin=0.5in(.\\
	A list of the variables we can set and their default values follows.
	\begin{verbatim}
		\evensidemargin
		\oddsidemargin
		\topmargin
		\textwidth
		\textheight
		\parskip
		\baselineskip
	\end{verbatim}
	
	\section{Tables}
	Sometimes we need to typeset tables.  For example, 
	consider Table \ref{animaltable}.  
	Any resemblance of the numbers in 
	Table \ref{animaltable} to those from any authentic poll is purely coincidental.
	\begin{table}
		\begin{center}
			\caption{\label{animaltable}
				Results from a poll that probably never happened.}
			\begin{tabular}{||l|c||}
				\hline
				\cline{1-2}
				\multicolumn{2}{||l||}{{\it What is your favorite animal?}}\\
				\hline
				Animal & Percentage of respondents\\
				\hline
				Dog & 43\%\\
				Cat & 44\%\\
				Schwarzenegger & 9\%\\
				We kill animals & 4\%\\
				\hline
				\hline
			\end{tabular}
		\end{center}
	\end{table}
	
	\section{Figures}
	We can also put figures into our latex documents.  For example, the
	image in Figure \ref{sniper} is found at 
	{\tt http://www.math.wsu.edu/kcooper/M300/sniper.jpg}, but must
	be converted to encapsulated postscript before it can be included
	in this document.
	\begin{figure}[ht]
		\begin{center}
			\caption{\label{sniper}
				Cats take vengeance on 4\% of respondents \cite{calvin}.}
			\vfill
		\end{center}
	\end{figure}
	
	\section{Homemade commands}
	\newcommand{\dickens}[1]{It was the best of #1. It was the worst of #1.}
	\dickens{ideas}
	\LaTeX\ was conceived as a programming language.  This is what makes it
	harder to process using a wysiwyg interface, but it also allows
	us to make our own shortcuts.
	If you know that a particular expression will appear repeatedly
	in your document, you can make an abbreviation for it, or even
	a command that allows you to specify arguments.  In our case,
	the sequence ``\dickens{{\it fill in the blank}}''
	is to appear many times in this section, so we created a command
	as follows:
	\begin{verbatim}
		\newcommand{\dickens}[1]{It was the best of #1. 
			It was the worst of #1.}
	\end{verbatim}
	This command takes one argument, which appears wherever a \#1 appears
	in the text for the command.  Thus, to typeset ``\dickens{examples}'',
	we need only to type \verb(\dickens{examples}(.
	
	\section{Citations and References}
	In technical documents there are many references to 
	typographical objects from the document, and citations
	of materials from outside the document.  \TeX\ \cite{knuth}
	and \LaTeX\ \cite{lamport} lets us keep track of those citations
	by name, rather than number.  Using the ``thebibliography'' 
	environment gives us automatic numbering of our references,
	while associating those numbers with names, so that we can
	refer to the references using the \verb(\cite( command.
	Likewise, for internal references, such as those to Figure \ref{sniper},
	we can assign labels to a counter using the \verb(\label(
	command, and refer to them using the \verb(\ref( command.
	
	\begin{thebibliography}{X}
		\bibitem{knuth} Knuth, D., {\bf The \TeX book,} Addison-Wesley, Reading, 1984.
		\bibitem{lamport} Lamport, L., {\bf \LaTeX: A Document Preparation System,}
		Addison-Wesley, Reading, 1986.
		\bibitem{calvin} Watterson, B., {\bf Homicidal Psycho Jungle Cat,}
		Andrews McMeel, New Jersey, 1994.
	\end{thebibliography}
\end{document}