\documentclass[12pt]{article}
\usepackage{amsmath}
\usepackage{amssymb}
\usepackage[utf8]{inputenc}
\usepackage[margin=1in]{geometry}
\usepackage{epigraph}


\title{
The Utopia Within \\
\large Clawing out the Utopian Impulse from the Human Psyche}
\author{Salvador Guzm\'an Jr.}
\date{August 16, 2023}

\begin{document}

\maketitle
\pagebreak

\begin{abstract}
    What is the utopian impulse?
    Why has the pursuit of this mythical creature bred such vicious monsters?
    And how can we tame reconcile this impulse with the nature of being?
    These are the questions that this essay seeks to answer.
    To be sure, these are not easy questions to answer.
    Many have tried and failed.
    I do not seek to add that growing list of failures.
    What I seek to do is to provide a framework for understanding the utopian impulse.
    More importantly, I seek to ameliorate this longing for utopia with the material conditions of our times.
    Surely, the utopia remains frustratingly out of reach without the proper tools.
    Yet, even with the proper tools, the utopia can create the next leviathan that will swallow us.
    Should we despair?
    I think not.
    I believe tha the utopia can be constructed from material tools.
    However, along this constructivist approach, we must defer concessions that are an unavoidable aspect of reality.
    We must accept that not all human impulses can be satisfied.
    Some impulses are those that seek to destroy the utopia.
    We will need to find a way to rid ourselves of these cannibalistic impulses.
    It is in this culling that we will find the utopia within.
\end{abstract}

\pagebreak

\begin{center}
    \epigraph{
        The lands we inherit from our fathers, were cultivated ere they were born, and yielded produce before they were cultivated. The products of genius are the actual creations of the individual; and, after yielding profit or honour to him, they remain the permanent endowments of the human race. If the institutions of our country, and the opinions of society, support us fully in the absolute disposal of our fields, of which we can, by the laws of nature, be only the transitory possessors, who shall justly restrict our discretion in the disposal of those richer possessions, the products of intellectual exertion?    }{\textit{Charles Babbage, Reflections on the Decline of Science in England}}
    \epigraph{
        The ascent to greatness, however steep and dangerous, may entertain an active spirit with the consciousness and exercise of its own power: but the possession of a throne could never yet afford a lasting satisfaction to an ambitious mind.
    }{\textit{Edward Gibbon, The Decline and Fall of the Roman Empire}}
\end{center}

\pagebreak

\tableofcontents

\pagebreak

\section{Introduction}
Human nature is a fact of life.
It is a fact that we cannot escape.
To some, this fact becomes a source of angst which motivates all sorts of rebellious behavior.
They accept this a priori fact as a challenge to be overcome.
The 20th century is replete with failed coups and rebellions.
Littered in the tens of millions are the corpses are the counterfactuals.
Just one death is example enough to protest against our utopian impulses.
One death is too many.
One act of coercion is one violation too many.
Any utopia that is not voluntary, peaceful and consensual is not a true utopia.

Now what do I intend to say in this paper with an introduction like that?
Surely, I have come to bury the utopia or to celebrate, with lavish malice, all its failures?
No.
No, I have not.

I have not come to praise the utopia since it requires no praise.
Rather, I believe that the utopian impulse is part of human nature.
It is a natural consequence of our physiology and more importantly, our psychology.
What I seek is demarcate palpably what constitutes viable pursuit of the utopia.
I wish to separate out the chaff of our efforts.
I do this so the future is not a repeat of the past.

The 20th century was a century of bloodshed.
It was a century of failed utopias.
I do not want more corpses and more violence.
Even some mild version of these violations is too many and testify against that particular attempt's vision of utopia.
Nay.
The utopian impulse must not be another excuse for violence.
God knows we possess them in abundance.
Thus, do I seek to outline a framework for the utopian impulse.
These are rules I wish to promulgate to the world.

The utopian impulse must be voluntary.

The utopian impulse must be peaceful.

The utopian impulse must be consensual.

The utopian impulse must be able to compete with the current state of affairs.

The utopian impulse must be honest and truthful.

Any violation of these, no matter now minor, completely invalidates that particular utopian impulse.
Applied to previous utopian impulsive regimes, we see that they all fail on some count.
The Soviet Union was not voluntary, not peaceful nor consensual.
Nazi Germany was not voluntary, not peaceful nor consensual.
Each regime can be viewed as an attempt, divorced from other attempts, to create a utopia.
And each attempt can be measured against the framework I have outlined.

\section{Characteristics of the Utopia}
How do I know what invalidates a utopian attempt?
Why do these rules matter?
Why should we care?

Through my learned experience, I have come to understand that the utopia is a state of being.
More importantly, I believe that experiences and the nature of reality is such that it can be nested within itself.
That is, the utopian impulse can be nested within the utopia.
In abiding by the rules of the utopia prematurely and even without the proper tools, we undertake construction of the utopia.
It is in our striving and our effort that we find a utopia within.
It is just a matter of universalizing our utopian tendencies and applying them to the world that we find the utopia.

Why should this be the case?
It is because reality and universal truth is recursive and fractal.
It does not suffer evil tendencies moderately.
It seeks to extricate them from the corpus of being.
Either they are contained or not.
There is no middle ground.
If they are contained, they a purging is necessary.
If they are not contained, then we have a utopia already.
This is because the utopia must be universal.
And to be universal it must adhere to certain rules of algebraic closure.
An attempt at utopia must be algebraically closed and containable within a larger utopia.
It is this incremental construction of the utopia that I seek to outline.

It is a constructible utopia.
This is the only type of utopia that is materially possible and should engross our idealistic efforts.

By sensibly engaging in utopian striving by espousal of utopian principles, even a failed attempt is utopic in practice.
Thus, can we plain state that no evil is too mild to be tolerated.
Any evil mars the utopia.
And any marring is prejudicial enough to completely invalidate an attempt.
In such a contingency, we must begin anew.
All the artifacts built up in such an attempt must be destroyed and torn asunder.
All of it.
This includes any power relations that exist in the attempt.
This involves any institutions, any bureaucracy and managers.
This is because the utopia is a state of being and the existence of members of a failed attempt will try to steer resources back to their failed utopia.
This is insufferable and should not be tolerated.

\subsection{Why a Constructible Utopia?}
In seeking to construct a utopia, we must be cognizant of the fact that we are not the first to attempt this.
We are rich with the experience of others and we should learn from their mistakes.
We should not repeat them.
It is enough to say that any death is too many.
It is not that a single death is a tragedy.
It is that all deaths are tragedies, never having their lyrical potential realized.
That a death has not a ode dedicated is a tragedy in itself.
All deaths are too much bear and chafe against the utopia.

It is all this and the fact that the utopian pursuit must be nestable within the utopia that I decided on this framework.
We are done justifying atrocities in the name of the utopia.
We know more know.
And we know that evil always begets more evil.
Treachery is not excuse for perfection and the only thing evil justifies is more evil.

May posterity learn from our mistakes and may us learn from the mistakes of our forebears.
In doing so, we achieve, however minor, the utopia.

Thus, to the question I pose in this section, \textit{Why a Constructible Utopia?}, I answer that no other option is possible.

\subsection{Voluntary}
Our modern world is a world of coercion.
Economic relations are often coercive and demand more than they give to the worker.
The worker is often left with no choice but to accept the terms of the employer.
This is a necessary condition of coercion.
The worker has no choice if they would like to eat tomorrow.

What my framework does so splendidly is that it follows a simple yet rigorous rule to compute whether a certain feature of the modern world is a feature in the utopia.
This will follow for the following sections as well.
If a certain features is missing from a modern institution, then it is not a feature of the utopia.

The exercise is the query: does institution $x$ have all features $y$?
If no, then $x$ is not a feature of the utopia.
Simple as that.

I believe that part of creating a utopia is invalidating what is not part of it just as easily.
It is a process of elimination.

Thus, in this section, I maintain that the utopia is voluntary.
Of course, one can easily see how the modern world is built on coercion.
One can expect many aspects of our daily life, such as managers and work, to be pruned away.
How this can be done I intentionally leave under specified.
It is up to an entrepreneurial spirit to figure out how to do this while staying within the bounds of the utopia.
You reader might be tempted to complain that my framework is too onerous to be implemented in reality.
To that, I respond that you do not possess sufficient imagination nor sagacity to ever be the vanguard of such an effort.

If you believe that evil is a necessary condition for utopian striving and that it presents just another tool, you have greater problems than this challenge.
Kindly step away from all political power because I believe you are a danger to yourself but mostly others.

How would such an attempt be implemented?
The idea is that the utopian striving will be enticing enough to attract people to it.
It will be a strength of this effort that others will recognize the utopian impulse and be attracted to it.
Much in the same way that evil repels, the utopian impulse attracts.

\subsection{Peaceful}
The utopia is peaceful and all attempts at utopia must be peaceful.
This is not optional.
This is not a feature that can be ignored.
Any violence at all defenestrates the utopian attempt into the pit of hell.
It must be discarded forever, all artifacts destroyed and all members of the attempt must be stripped of their power.

Why must this be the case?
That is because the way we currently see the world organized is through the recursive structure of managers.
If any violence is allowed at any level of the managerial apparatus justifies its use at all other levels.
And thus will their workers be subject to violence.

It is intolerable.
It is intolerable that any violence be allowed.
Any special pleading to the contrary betrays the guilty conscious of the supplication.


\subsection{Consensual}

\subsection{Competitive}

\subsection{Truthful}

\section{Utopian Typology}
\subsection{Universal Utopia}
\subsection{Negative Utopia}

\section{The Utopian Impulse in the Modern World}

\end{document}
