\documentclass[12pt]{article}
\usepackage{amsmath}
\usepackage{amssymb}
\usepackage[utf8]{inputenc}
\usepackage[margin=1in]{geometry}
\usepackage{epigraph}

\title{
Aggression as Behavioral Iteration \\
\large A Comprehensive Theory of Domain Specific Aggression }
\author{Salvador Guzm\'an Jr.}
\date{June 29, 2023}

\begin{document}

\maketitle
\pagebreak

\begin{abstract}
    Long has behavior been studied as a indeterministic process.
    I believe this to be not for simple lack of theoretical underpinning.
    A certain degree of Romantic affection has blinded our analytical scrutiny to pursuing such an end.
    Given that I am properly addressing this established circumstance, I wish to remedy this condition.
    Whatever resonance once possess for the human condition, it should not distract from scientific and analytic regor.
    With the Romantic vision of humanity discarded, we can understand behavior and behavior changes as a simple Mathematical relationship.
    This paper will attempt to establish a mathematical theory for behavioral iteration.
    The mathematical theory underpinning human behavior and all animal behavior rests on well-understood axioms of measure theory and probability theory.
    With this in mind, I will attempt to establish a mathematical theory for behavioral iteration.
\end{abstract}

\pagebreak

\begin{center}
    \epigraph{
        The lands we inherit from our fathers, were cultivated ere they were born, and yielded produce before they were cultivated. The products of genius are the actual creations of the individual; and, after yielding profit or honour to him, they remain the permanent endowments of the human race. If the institutions of our country, and the opinions of society, support us fully in the absolute disposal of our fields, of which we can, by the laws of nature, be only the transitory possessors, who shall justly restrict our discretion in the disposal of those richer possessions, the products of intellectual exertion?
    }{\textit{Charles Babbage, Reflections on the Decline of Science in England}}
    \epigraph{
        The ascent to greatness, however steep and dangerous, may entertain an active spirit with the consciousness and exercise of its own power: but the possession of a throne could never yet afford a lasting satisfaction to an ambitious mind.
    }{\textit{Edward Gibbon, The Decline and Fall of the Roman Empire}}
\end{center}

\pagebreak

\tableofcontents

\pagebreak

\section{Why not A Mathematical Theory for Behavioral}
This author cannot help but be impressed at the scant attention paid behavior.
It is often the sensibilities of our times that establish the topologies of allowable study.
The study of human behavior is a topic that is of great interest to many but no particular interest to scientist.
I venture the position that this is because the study of humanity proper as a biological species offers to intimate a view of the human condition.
There are things we often would care to omit from our understanding to protect our fragile view of the world.
This author admits to no exceptions in their own case.
However, I believe that intellectual maturation necessitates examining the world as it is, not as we would like it to be.
And if one were to seek to change it for the better, one better have accumulated a formidable body of knowledge first.
As an intellectual, I believe that two great sins any thought-minded person can commit is changing what they don't understand.
The second is refusing to countenance aspects of reality they find disagreeable and preventing others from doing the same.

With all that said, we can begin on our ambitious adventure to describe behavior as a mathematical process.

\section{A Mathematical Theory for Behavioral}
The idea that human behavior can be understood as a mathematical process is not novel.
Yet it stands as one of few pillars of modern life that has resisted attempts to standardize and formalize.
I feel this is more due to our naive sensibilities than any particular theoretical hurdle.
Perchance there exists room for doubt that such a thing was possible until a couple of centuries ago.
Before probability theory and measure theory, it would have been very difficult for such an advent to even start.
Given that we have been met with these elegant tools, no such excuse is possible now.

As I have intuited before, I associate the relative naive state of this field of study to sensibilities
The prodigious explosion of theoretical tools produced in the last two centuries and their judicious application to domain problems testifies to their success.
That they have not ventured to a sensitive topic and rendered out God from the process says more about us than our tools.
I believe that the time has come to remedy this situation.

What I propose is a mathematical theory for behavioral iteration.
I wish to put forward a theory that can describe behavior to a certain degree of certitude given the resolution of our tools of analysis.
I believe that human behavior can be understood as the sum of sources of behavior.
What this entails is that there biological sources of behavior that ebb and flow in strength and frequency.
The environment and biological cues are what decide the strength of the call for specific behaviors.
Thus, one can see the body as a holding cell for a multitude of behavioral impulses.
These impulses "fight" for supremacy and monopoly over the modalities of action.
That is, because of the scant morphology of the human body, mental focus cannot be so easily divided.
Thus, the body is a battleground for the impulses of behavior, each vying for monopoly.

\subsection{An Expedition into Behavior}

The relative strength is affected by multiple factors. The most noteworthy are,
\begin{enumerate}
    \item Time of day (for diurnal animals)
    \item Relative biological premium (as relating to homeostasis)
    \item Bounds of modality (lower modality behaviors are more less likely to experience competition)
\end{enumerate}

\subsubsection{Sleep}
A few words on the above.
The time of day is a factor for animals that experience sleep.
Sleep is time-sensitive in diurnal species and thus the relative strength is controlled by time-sensitive factors.
In the case of humans, the circadian rhythm is the most important factor.
It incorporates the presence of light, which is assumed to be sun-derived.
The presence of melatonin moderates the strength of the sleep signal.
The relative biological premium is a factor that is related to the homeostatic state of the organism.

\subsubsection{Biological Premium}
As the last sentence alludes, some behaviors have a biological premium associated with their activation.
Food is the most obvious example.
It is a metabolic imperative that an organism satisfy their nutritional requirement to continue living.
This is not negotiable.
And thus, the impulse strength to consume food will be very high.
Of course, that does not mean the strength of this signal will not be the same over all conditions.
When an animal is near starvation, the signal will be very strong.
When an animal is satiated, the signal will be very weak.
Thus, food and consumption thereof is the sink that the drive to eat advances the creature towards.

The satiation of this signal can be seen as the level of scarcity of food.
The more abundant food is, the more likely the animal is to consume it and stay satiated.
The more scarce food is, the more likely the creature will expend energy in search of food, under the influence of the drive to eat.
The satiation of the drive can be seen as an external event that triggers a fall in the relative strength of the drive to eat.
In the calculus of power, acceding this particular behavior leads to competition among the remaining impulsive power-brokers.

\subsubsection{Morphological Modality}
The morphology of an organism is necessarily limited.
In this discussion, morphology does not just refer to the physical shape of the creature.
It will also refer to the dedicated manner in which the creature achieves homeostasis.
For example, humans are bipedal and thus have access to hands while standing.
This is a modality that most primates do not have access to.
Impulsive and behavioral access to what the hands are doing is a modality that humans will experience.
The same can be said for anything that is available in relative abundance.
Attention is also a modality that is available in limited supply.
We can talk about splitting focus happening at the expense of the quality of the dedicate focus.

\subsection{Abstract Discussion}
As of now, I have ambiguous whether I am referring explicitly and narrowly to humans.
I wish to continue speaking in this manner.
Note, that what I am stating is general and applies to any organism.
Obvious, the specific case of humans commands our attention.
Just keep in mind the general nature of the statements.
The mathematical machinery is rather abstract and general by design.
Thus, do I hope the reader forgives any trepidation experienced when applying this model to the non-human case.
In the future, discourse will have to cater specifically to other animals.

Because of the relative abstract nature of this discussion, I will not delve into the biological mechanism by which behavior is achieved.
Quite frankly, it does not matter for the extent of this discussion.
All that matters is that once behavioral cues are posited, we can understand their attenuation in respect to competing behavioral cues.
This analytical theme is similar to how probability theory can be studied.
One can learn the genesis of probability or the axiomatic manipulation of them.
Or both.
Given that I am not interested in the genesis of behavior, I will not discuss it.
My discourse will be circumscribed to the understanding of the dynamics of a set of given behavioral impulses competing for actuation of some modality.

\section{What is Behavior?}
From all that has been thus far, it should be clear what behavior is.
To summarize, behavior is the sum of all the biologically-sourced impulses of an organism as well as all their accessible morphological modalities.

\subsection{Confronting Modern Sensibilities}
Note that I did not reference the phenomenology state of the creature.
I fear modern sensibilities take exception with any science that not incorporate phenomenology
To those who feel accordingly, I will say what Laplace said to Napolean when question on why his mathematical machinery made no reference to God.

\textit{Je n’avais pas besoin de cette hypothèse-là.}

If exhaustive reference to the phenomenology of the creature would have enriched my theory, I would have included it.
As of now, it can only impoverish it.
I do not wish to always delve into the attitudes and mores of the present but must do so when they hamper intellectual exploration.
It exquisitely frustrating to see potential avenues of research thwarted by the imprudent non-intellectual public.
I will be so bold as to say that the public is not only non-intellectual but should effectively be excluded from intellectual discussion.

However, this particular theory is not special in this regard.
I fear that any theory that seeks to reduce human behavior to a rational calculus (which it is) will suffer public opprobrium.
Opportunistic-minded public speakers will happily and wantonly whip up a frenzy where intellectual discourse will suffer and paint their victory as elevating the human condition.
Nothing could be further from the truth.
In fact, not discussing the theory underpinning human behavior has let to all manner of gross exercise of power.
Communism, for example, rested its intellectual foundation on a false understanding on human behavior.
The repulsive attempt at utopia leads to the wholesale slaughter of millions.
As of now, 2023, there is yet no apology for the crimes of communists.

\subsection{Behavioral as Biological}
In light of the discussion on genetic determinism, it should be clear that behavior is biological.
It is not a question of whether some behavior is genetic or not.
All behavior is genetic in origin.
Perhaps, if I were to bestow them a gift of charity, the discussion is referring to what specific environmental cues trigger what behavior.
Ultimately, all behavior is biological.
It makes no sense to say that the stock of behavior for a creature is environmental.
All behavior rests within the creature.

\subsection{Genetic vs. Phenotypic}
A more nuance affordance would be what is the relative contributions of the genetic and phenotypic factors.
The genome of the creature is the blueprint for building the creature and an ongoing reference for metabolic genesis.
The phenotype of the creature is a sort of projection of the genome.
It is instantiated to a particular environment.
The phenotype will always be a subset of the genome since not all genes are actively expressed all the time.

The way I model it is the genome is the source code.
The source code of a program consists of all comprehensive instruction set for all conditions and states.
The phenotype is the program at run time.
The particular running state of the genome is the phenotype.
In this model, it is obvious that not all conditions are met and thus not all instructions are executed.
Only when certain conditions are met, will certain instructions be executed.
This simple model yields great insight into the nature of the genome and phenotype.
When it is claimed that something is genetic, they are referring to the phenotype, which again, is a subset of the genome.

\subsection{Genetic vs. Environment}
As referenced before, there is no thing as environment-sourced behavioral repertoire.
All behavior is genetic in origin and acted out in a phenotype.
The full set of a possible phenotype a creature can instantiate defines the bounds of all possible behavior.
The environment cannot add to this set.
In fact, not only is this view spurious and erroneous, it mixes the direction of consequence.
Since genes are the ultimate source of all behavior, they "allow" the environment to trigger and differentiate between appropriate behaviors.
The environment can "choose" what phenotype is active only because the genes allow for this "choosing."
Forgive what may seem a tautological discussion.
Modern sensibilities have so thoroughly confused the issue that I must be explicit.
I cannot help but feel that someone has tampered so thoroughly with language that is now near impossible to extricate a proper discussion on the subject.
Again, I cannot help this is by malicious design.

A creature born with a limited genome can only act out and behave in so many ways.
Behavior is not infinite.
It is bounded by the genome.
Modern discussion on human adaptability and plasticity overemphasizes the size of our behavioral repertoire.
The behavior we see in modern humans can only be limited.
It is the novel application of a limited set of primitives in brilliant ways that distinguishes humans.
Just keep in mind that this sort of "grammar of behavior" is not unique aspect of other humans.
Why humans were so particular successful in this regard, I will muse over next.

\subsection{Extended Phenotype}
The extended phenotype is a concept, popularized by Richard Dawkins, is illuminating.
The idea is that the possible phenotype at any present moment will be limited by the genetics and environment.
This necessitates concessions to be made as to which condition affecting homeostasis to address with priority.
However, the limits of the genome and the phenotypes need not be ultimate in practice.
For example, hermit crabs can address the issues of safety from predation by investiture in a shell.
The shell is not part of the crab's phenotype and you will find no direct genetic corollary for it in the genome.
Thus, the shell is part of the extended phenotype of the hermit crab.

Humans have taken the concept of extended phenotype to the extreme.
The size of the human brain is a testament to the "choice" made by nature to invest in the potential of the extended phenotype.
What humans lack in relative physical weakness, they make up for by investing in technology.
Anything from clothing, shelter, tools, and weapons are all part of the extended phenotype of humans.
This arsenal against the conditions of nature help moderate the local environment to more appropriate conditions to maintain homeostasis.

\subsubsection{Bounds of Extended Phenotype}
What determines the bounds of the extended phenotype?
Simple.
It is the obligatory conditions of homeostasis.
Homeostasis is required because it permits the creature to live long enough to reproduce.
However, as was the case with humans, the potential for capitalizing on an extended phenotype requires a certain degree of biological equipment.
The human brain is the most exaggerated example of this.
Organisms do not possess the same neurological abundance as humans.
That is why their extended phenotypes are always found to be more simple.
That said, do not let our own become sinecures of extravagance.
They work the same way our own do.
They offer advantages that the creatures were not natively born with.

The extended phenotype is thus shared with all members of a species.
Given this, it becomes the bedrock for any cultural transmission.
One can get a cursory glance at how culture genesis may work.
But that is a discussion for another time.

\section{Mathematic Appraisal of Theory}
At this point, the model for behavior is sufficiently developed.
It is the anarchic chorus of impulses competing for expression.
The drive for sleep, food, mating, diversion and so on are regulated by the environment and also compete with each other.
While the environment is an important factor, it cannot be said to be the exhaustive regulator.
The genome is the ultimate source of the general strength of each behavioral impulse.
For example, it dictates that the drive for food is stronger during famine.
But also in general, the drive for food can be said to be stronger than the drive for diversion in general.

The theory of behavior rests on considering each impulse as a probability.
While the introduction of probability invites a certain array of interpretations, I already have one explicitly in mind.
It is not that I am assigning a probability to each impulse, although such an interpretation can be made rigorous.
Rather, I am considering the sum total of the strength of all impulses equal to 1.
Each impulse is a fraction of the total, vying for dominance.
The dynamics of this system describes the day to day behavior of the creature.

Now at this point, we are confronted by multiple forks in the roads.
I described a very primitive mathematical for behavior.
I will describe the possible choices we have for scaffolding forward a more nuanced theory:\

\begin{enumerate}
    \item An impulse must achieve a certain arbitrary threshold shared with other impulses before it is acted out.
    \item An impulse must achieve 1 before it is acted out.
    \item An impulse must achieve a certain arbitrary threshold unique to it before it is acted out.
\end{enumerate}

Out of all of these, the third is most tractable if only because it is the most ambiguous.
However, I would prefer to avoid it if possible.
I do not want to hunt around for the biological threshold.
I will leave that to someone more biologically minded than I am.

The first then is the one we will choose.
We can set an arbitrary threshold for all impulses.
Keep in mind what we set as the threshold will have to reconcile with the relative growth rates of each impulse.
For example, if we know the fastest rate of impulse strength growth to be the drive for food under famine conditions, all other impulse growth rates must respect that.
Such is the mathematical minutiae that must be considered when developing a theory of behavior.
I'll spare my audience a more detailed discussion of the mathematical machinery.
Just know there is one.

\section{What is Aggression?}
\subsection{Aggression as a Strategy}
\subsection{Aggression Tolerance as Contingent on Personality}

\section{What is a Strategy?}
\subsection{Strategies For an Iterated Game}

\section{Aggression as Pursuit of Winning Strategy}

\section{Summary}
This paper constitutes my attempt at an invitation for others to learn more about their enslavement to the managerial class.
I have tried to be as comprehensive as possible in my analysis.
At the same time, I know this paper will be found wanting and of limited scope.
I invite readers to undertake their own research and to share their findings with the world.
Only when knowledge is free, will men be free.

\subsection{Behavior as a Mathematical Process}
\begin{enumerate}
    \item 
\end{enumerate}
\subsection{Aggression as Strategic Iteration}
\begin{enumerate}
    \item 
\end{enumerate}

\section{Suggested Points of Departure for Future Research}

\end{document}
