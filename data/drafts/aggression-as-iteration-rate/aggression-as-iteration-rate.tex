\documentclass[12pt]{article}
\usepackage{amsmath}
\usepackage{amssymb}
\usepackage[utf8]{inputenc}
\usepackage[margin=1in]{geometry}

\title{
 Aggression as Strategic Iteration \\
 \large The Mangerial Class and the Exploitation of the World}
\author{Salvador Guzm\'an Jr.}
\date{June 24, 2023}

\begin{document}

\maketitle
\pagebreak

\begin{abstract}
    While popular discourse has focused on the role of the capitalist class in the prolonging of certain deprivations, the managerial class has escaped all culpability.
    This paper seeks to rectify this oversight by examining the rapacious tendencies of the managerial class and its role in the exploitation of the world.
    Exploitation is interpreted rather conservatively in this paper.
    Frankly, it is not needed.
    There will be no resorting to Marxist definitions to extract sensible attitudes to disposition one against the managerial class. Instead, the paper will focus on the managerial class, what they are, how they came to be, the consequences of their unattenuated power, and how to dispose oneself against them.
    You will find no panacea here; only sagacity and wisdom of the analytical word. What motivates this author to write this paper is the envisioning of a better world beyond the managerial horizon.
    A world where the resources of the world are not held hostage by a hostile, callous and neglectful interest.
    This can ours, reader.
\end{abstract}
\pagebreak
\tableofcontents
\pagebreak

\section{The Manager as the Apex Corporate Predator}
A tragedy in the making can be spotted in the manner by which private interests can define the contours of popular discussions.
While the modern imagination is still hauntec by the trauma of the 20th century, there lurks a class of men and women, more detestable by action, than communists or capitalists.
I am of course referring to the managerial class.
Whatever be the country's specific ideological affiliation, it produces a niche for the same group of men that hazard an opportunity to sate their ambition.
While the particular job description of the particular manager may vary, what does not vary is their outsized influence.
This influence extends beyond the narrow confines of the workplace and bleeds into the public sphere.
And bleed indeed as the managerial class has shown themselves to be a rapacious bunch.
In exercise of their power and appraisal of their own interest, they have consumed the world like no class before them.
To safeguard their longevity, they have erected a predatory system of control that is as insidious as it is pervasive.
No industry is safe from their influence nor extends beyond their narrow discretion.
Thus, it is the narrow description of their particular profession, instantiated many times over, that lead to their proliferation.
It is the sincere opinion of this observer that the dysfunctional state of political course is a product of their outsized influenced.
As a class, they respond to local and greedy incentives that maximize their own utility.
And because of the unbounded nature of corporate power, does the tendrils of their influence extend to the world.
It is the purpose of this paper to examine the managerial class, their role in the exploitation of the world, and how to dispose oneself against them.

\subsection{Capitalism vs. Communism: The Perfect Distraction}
The 20th century was a century of ideological conflict.
The two dominant ideologies of the time were capitalism and communism.
While this is a gross butchery of the conflict, it serves well our cursory glance.
What we become interested in is the managerial experience and conduct in both ideologies.
\subsubsection{Capitalism}
In capitalism, the managerial class is a product of the separation of ownership and control.
The separation of ownership and control is a product of the industrial revolution.
A more detailed discussion on the cause of the managerial class within a capitalism awaits us.
For now, we must understand that the separation of ownership and control was necessitated by the relative complexity of industrial enterprise.
In this paper, capitalist is understood as the owner of capital and nothing more.
No deragatory connotation is implied.
No longer could a single capitalist manage the affairs of the enterprise.
Rather, they were forced to delegate away the responsibliity of management to another group of people.
The more the capitalist owned, the more the daunting task of management became reality.
It was simply not feasible to supervise all of productive work of their capital.
As a class of people, capitalists choice to delegate to a proxy the problem of management.
This event is what dawned the separation of ownership and control.
And in this separation do we find the genesis of the managerial class.
Because of the unbounded nature of management, the managerial class was allowed to grow, often at the expense of other workers.
They became de facto owners of the enterprise by virtue of the power they exercised.
And it is in this power that they reveled.

\subsubsection{Communism}
The managerial experience differed very little in communism.
While certaintly the profit motive shrank in importance, the need to manage large enterprise did not.
As can be observed in the many cases of practiced communism, the managerial class was well regarded.
Large enterprise would always be the bedrock of their necessity and in the absense of capitalist, their respective power could only grow.
To say that local politics of day to day management are what shaped the manager's intuition is a gross understatement.
In fact, competition between managers was the norm.
And competition between different regions characterized the political landscape.
A common trend you will observe is that in competing with each other, they inadvertently expand their domain.
This is a common theme in the managerial experience due to the unbounded nature of management.
Thus is the vampire of the managerial class able to transcend their wicked fangs over all of humanity.

Do note that lackluster analysis in this section.
That is by intention.
This paper is not about communism.

\subsubsection{Fascism}
Fascism is a managerial ideology.
That should be self-evident from common parlance related to fascism.
What should also lead one to suspect the manegerial nature of fascism is the emphasis it places on the management experience.
The way a manager runs their respective workplace is like the perfect fascist society.
The manager is the dictator of the workplace.
This should not in any way be a surprising conclusion.
The common experience that many Americans workers have with their respective managers is one of a petty tyrant.
The experience is commonly negative because of the unbounded nature of management.
While Nazi Germany will evade our analysis, proper examinations of the role of managers in industry and politics exist.
The reader is encouraged to read these and see how the death of the profit motive, which also existed in Fascism, did not preclude managers.
Rather, it was the perfect environment for them to thrive.
They are the peak corporate parasite of all industry.


\section{Motivating Musings over the Managerial Experience}
But why muse over the managerial class?
Certaintly, would an exhaustive analysis of the managerial class reveal anything novel?
I think so.
Beyond such a speculative impetus, analysis of the managerial class has suffered from lack of development compared to other institutions.
This is evident from the lack of public consciousness over managers as a class and commensurate understanding of their influence.
Common discussion that prevail tend to overemphasis the role of the capitalists and diminish the role of managers.
While it is natural to fail to appreciate this role as most people's understanding of a manager is exhauted by their respective profession, it does not tell the full story.
Surely, the manager class has partipated in shaping the contours of the public consciousness.
And in this shaping, they chosen, whether intimately conscious of their endeavor or not, to obscure their own role.
Their obscurity is well known.
One imagines a similar situation occurred in the late 19th century with the rise of the capitalist class.
The rise of a new class of power brokers is always traumatic to those they discolate.
And the managers are no different.
The capitalists would at times spend large sums of money to propagandize their own role and rehabilitate their image in the public eye.
Preferred to this however would be obscurity.
Ultimately, the capitalists have no convincing argument motivating why their power should continue to accrue unabated.
And such is the situation that we find with the managerial class.

While the capitalist class has been well studied, the managerial class has not.
This is by design.
The capitalist class was eventually forced by circumstanced to reveal themselves, their position, their power and their ideology.
The managerial class has not faced any sort of reckoning yet.
I must remind the reader that the promulgation of the capitalist view of the world needed to be cajoled out of their exponents.
Such theory was not forthcoming without bringing the conflict to the class proper.
And will the managerial class be any different?
No.
It is this fight that I wish to prepare you for, dear reader.

\subsection{Rise of the Managerial Class}
The previous explication should have established the genesis of the managerial class.
More important, with any new class of people that ultimately define the contours of power, the cause of their rise is of paramount importance.
The managers did not create themselves, ex nihilo.
The social conditions that preceded their rise are what ultimately ensconced a world with managers.
Once a critical mass of managers were in existence, the tired and pervasive laws of power took over.
Given the unbounded nature of management, from the small initial cliche of managers instantiated across industry came the rise of the managerial class.
From there, the algebraic laws that constitute the calculus of power demanded that managers be delegated more and more responsibliity as industry and enterprised matured.

Let us examine some of the aggravating factors that led to the rise of the managerial class.

\subsubsection{Separation of Ownership and Control}
With the rise of industry, the laborial demands placed on the capitalist class grew.
No longer could a single capitalist manage the affairs of the enterprise.
Industrial enterprise had reached a point of maturation where the task of management was too daunting for a single person.
Thus, the capitalist was not able to be a capitalist and a manager.
One of the tasks had to be delegated.
And it was the task of management that was delegated.
By virtue of this arrangement, the capitalist could maintain their ownership of the enterprise while delegating the task of management to another group of people.
And here do we find the genesis of the managerial class.
In preferring to be a capitalist (to retain their commensurate power), they chose to create a class of people.
This class of people would be the managers.
This simple act not only authored into existence a class of men but also separated spheres of concerns.
It is this divorce that created a role for a power-conscious elite to ultimately capture.
And capture they did.

The ultimately motivating factor for the capitalists was the desire to retain their power.
This by itself is not a novel observation.
All creatures, simple or complex, seek to retain their power.
What makes the case of the capitalist unique is that the rational calculus of power demanded that they create a new class of people.
To retain their power, they needed to dissolve the constraints faced by enterprise.
They could not feasible be managers and capitalists.
The demands on their time would be too great for any person.
On a more fundamental level, the capitalist class would not be able to enjoy the fruits of their labor.
It is not that capitalists abhorred work.
They were happy to work to build up the incipient level of complexity in the first place.
In their delegation, they shirked laborial duties and happily retained ownership.
They viewed their ownership as safeguarding their power even though a proxy would run their enterprise.
And should any conflict arise, they esteemed that their de jure ownership would be suffice to reconcile any conflict with management in their favor.

Of course, the capitalist did not envision that their rational and locally-greedy choices would create a new class of men.
They did not envision that they would lose influence over the affairs of their enterprise.
Again, their ownership was supposed to be a safeguard against this.
However, because of the calculus of power and the unbounded nature of management, the capitalist class was ultimately dislodged from their position of power.
The managerial class would ultimately capture the enterprise over decades as more and more of industry was "managed."
The capitalist class experienced a "hollowing-out" of their influence as the managerial class grew in power at their expense.
This happened because of local decisions.
The capitalist class did not envision that their local decisions would have global consequences.
In the end, the opportunities that enticed managers into existence offered them opportunities to exercise their volition.
By virtue of management, their volition was tied to management.
And by managing industrial enterprise, they were able to exercise discretion to their own benefit.
Again, by virtue of managing, the will of the manager is interpreted as the will of the enterprise.
And thus was managerial discretion the helm of industry.

\subsubsection{Managerial Discretion as the Rapacious Tyrant}
A beguiling question is why did not the capitalist class see the possible hazard of managerial discretion.
That their own discretion would be usurped in the process of delegation surely was a possible contingency.
And that the enterprise would succumb to managerial capture did not evade their notice.
The profile inviting analysis of the manager is even more perverse than how the situation started.
It is 2023 and even now, the discretion of the managers has not been fully realized by the public, let alone critiqued.

Owner or not, safeguard of the capitalist's de jure ownership was never in jeopardy.
Perhaps the capitalists had a blind spot that emphasized the legal fiction that protected their property rights.
Perhaps they did not value de facto discretion if they could ultimately resolve any conflict by resorting to their legal narratives.
Whatever the reason, it can be understood that the capitalist either overplayed their legal obligations or did not value discretion as practiced.
Slowly the effect that the capitalist had as a manager eroded because they let it erode.
And filling the vacuum became a natural proclivity of the new class of managers.
That they greedily feasted on the opportunity to become intoxicated with their own discretion is not the story.
The novelty is that the same local decisions made by capitalist all around the world but mostly in countries with advanced enterprises lead to the instantiation of this class.
There was no coordination between the capitalists.
Their own greed was the guiding light that rowed them to their own perdition.


\subsubsection{Death of the Profit Motive}
Rationality and the power motive succumb to the same calculus.
That some rigorous theory undergirds the exercise of both should not startle any intelligent person.
All things are knowable and all things are predictable in the abstract with a robust enough theory.
And power, money and industry are no different.
Celestial mechanics and the wonderful creatures that inhabit it are no strangers to the terristrial firmament that binds their calculus.
And even God must succumb to theory.

If some adversarial party were to attempt at elucidation of the calculus at play of the managers, what would they have to defer to?
What externally accessible evidence exists to support the existence of this managerial class?
Aside from the astute socially available observations that testify to a like-minded and interested party, we have the death of the profit motive.
The profit motive is the most salient and obvious evidence of the existence of industry and capital at work.
Naive intution would inform us that profit is necessitated by avaricious churning of the capitalists.
Without being able to realize a return on their investment, folklore states that the capitalist would dissuade exposure to their capital.
And yet, the profit motive is dead.
How do we reconcile this?
Did the capitalists turn into philanthropists?
Do they donate their capital out of the kindness of their heart?
No.

The profit motive is dead because the managers have no direct incentive to realize a profit.
A prolific incantation of their professional creed involves them managing.
Nothing else.
The hearty implication is that they will manage in proxy of the capitalists and that they will manage in the best interest of the capitalists.
And if that not realized, they can be removed from their position by the latter.
In absence of heavy handed intervention, they will manage how they see fit.
Any attempt by the capitalist to infuse the enterprise with their own discretion will be interpreted as malicious micromanaging.
In fact, the word micromanaging is quite telling since the managers proper consider interference in their jurisdiction as being beset by a petty "little manager."
It is assume by allusion to their respective job title that the manager rules in all things management.
Hence it being no little understatement that the manager must be a fascistic dictator.

While a singular manager can be fired, a class as whole is not so easily removed from their volition.
Managers will manage, regardless of their individual idiosyncrasies.
And maybe if the capitalist, the owner of capital, were the sole interest that the management deferred to, firing would stem the managerial tide.
However, a modern enterprise, with its customers, employees, suppliers, board of directors, and shareholders, contains diffuse interests.
The capitalist is no longer alone.
This was the faustian bargain that the robber barons of old contended with in order to expand the domain of their industrial undertaking.
Choose to retain full voting rights of their enterprise and limit their growth.
Or expand their industry and dilute their influence.
The latter was chosen often enough for it to be the norm.

In a diffuse settings, the appearance of serving the interests of the capitalist is sufficient.
To the extent that they accomplish this only goes toward establishing their competence.
Only outright malicious or incompetence ends the reign of a particular manager.
But management as a theory of industrial praxis never recedes from this earth.
The profession category is never eroded.

Thus, given that all has been said, where is the profit motive?
It appears peculiar that any particular occupation may escape the profit motive.
But if there were to be one, it would be management.
Management is the most removed from industrial output.
They do not directly contribute to production quota but by virtue that the sites of production continue unemcumbered.
Anything above outright atrocious failure is success for the manager.
Hence they are protected from having to manage a profit into existence.
Profit is not relegated to their purview but to the capitalist.
Any profit realized thus is a boon to their credit.
If the manager does not realize a profit, they are penalized because it was never their responsibility to do so.
While this may seem innocuous in the local sense, at large this is the knife in the back for felling the managerial class.

\subsubsection{Managerial Job Security}
Speaking of felling, who manages the managers?
Another manager of course.
Marvelous.
Enterprise as a hierarchy of managers is management run.
And given that all entrepreneurial type of people are extremely conformists, eventually management bleeds into everywhere.
Outside from the direct stakeholders of any enterprise, managers are allowed to act with near impunity.
In fact, the only people managers have to answer to are other managers.
The managerial advantage in job security is real.
There is even a practice of measuring the competency of employees called annual reviews.
Who by sheer chance do you think performs these reviews?
The managers of course.
Do employees underneath a manager have any say in the review of their manager?
Do employees get to review their manager in turn?
No.
And when the manager proper is reviewed, who do you think performs the review?
Another manager higher up in the hierarchy.
It's managers all the way up.
This is the rule by manager alluded to earlier.

\section{The Exploitation of the World}
What is established now presently is the sheer unparalleled and unabated power of the managerial class.
They exist, the rule, they accrue power.
Reader, should you disagree with any of the previous points, I have failed in my exposition.

How is it that this class exploits?
What are the contours of their exploitation?
Who is the most exploited?

\subsection{The Exploitation of their Workers}
Surely, anyone who is not a manager is exploited, including the capitalist.
As I hinted before and nominal studies have established, the manager's pay is not commensurate with their contribution to the enterprise.
In fact, given the poorly specified job description of the manager, their advantage in the industrial context depends on their poor specification.
The sine qua non of the manager is the ambiguity.
Instead of having a assemblage of worker profiles instantiated for work, the manager has the right of discretion to exercise what needs to be done.
And thus can they check the ambition of any other class of workers vying for their power.
A power-conscious class will be able not only to maintain their relative position but enhance it when contingency provides.
As mentioned before, the ambiguity coupled with their innervating volition is purchase enough for managers to afford the balancing act in regards to power.
No malice nor violation of managerial norms is necessary.
The workplace becomes an arena of power passively waiting for the manager to assert themselves.

\subsection{The Exploitation of the Capitalists}
While capitalists may own the means of production, someone has to manage them.
Because of the separation of the ownership and control, the capitalist are at best a passive influence on the enterprise.
More active is the managerial iterative churning of their will.
Hopefully, I have established how their's is the one that matters.
Given that they actively work so that the capitalist does not have to, they become more familiar with the enterprise.
And given that they supposed work in the best interest of the capitalist, this situation is tolerated by the capitalist.
It is with this that we find the scenario where the manager can justify any action by claim of being germane to the interest of enterprise.

If you think this is not safeguard enough for the manager against any skeptical capitalist, bear out the thought experiment.
Remember, a competent manager is at least an equal to the capitalist and a superior industrial counterpart as far as the workplace is concerned.
The latter will not be intimate of any knowledge that the manager has not already rinse of actionable interest.
That a capitalist, by virtue of a legal claim on capital, can know more of a workplace he has not frequented is dubious.
Thus is the master of production subjugated to the manager of production.

\subsection{The Exploitation of Capitalism}
While a comprehensive definition of capitalism can be quite illusive, here we will resort to simple themes for our purposes.
Capitalism is a economic and financial regime predicated on the illustrious defense of private property and its judicious exercise in pursuit of returns.
We have already endured and surpassed the twilight days of capitalism.
What has replaced it is the managerial regime.
In such a regime, the prospect of returns retains some of its initial primacy but has withered in practice.
What is more important is the wanton whims of the managers.
If profit is realized, it is by no incidence the grace of managerial spectacle.
Thus, the version of capitalism that the managerial class has wrought is a bastardized performance of the original.
Their interest as a class transcends and subdues the profit motive.
If profit can be realized, it better be to the advantage of the manager.
This coupled with their own ability to anchor their own restitution inevitably leads to a surplus of capital they experience as wages.
This can be readily observed that for ever job profession, the respective managerial flavor always commands a premium.

It is my assertion that this premium that managers enjoy in their salary and job security is not by virtue of their commensurate contributions to enterprise but by the poverty of their job description.
The miscellany of their job description is the source of their power.
Any industrial responsibility they wish to retain for themselves can conducively be done so.
Any responsibility they wish to shirk can readily be delegated away.
Any choice they make will be celebrated upon exhumation.
They can't lose since any execution of volition will be considered managerial.

It should become obvious how having a surplus of funds spinning in the rotisserie of managerial intrigue is the subversion of capital.
Their compensation is effectively guaranteed to be of surplus value and thus do they evade the profit motive.
There is nought a capitalist can do but to slavishly acquiesce to appease the manager who now holds their enterprise hostage.

\section{Communism: Distraction by Design}
This author cannot be amused by the sudden upsurge in support of communism.
The more honest of liberals will not be able to deny that there has been an uptick in socialist thought among the young.
This is of course by design of the propaganda class but has been propped up by the managers as well.
As mentioned previously, communism is not immune to managers.
In fact, given the centralized experienced that socialist economies exhibit, managers will have even more power in socialism.
The push towards socialism, or the road to serfdom as Hayek would have it, is happening because the managerial class has given it it's stamp of approval.
While the manager class has not directly created the formidable propaganda aparatus for communism, they see value in its existence.
Managers can't lose with communism.
The propaganda class is full of disaffected Marxist agitators and politically adjacent agents.

\subsection{The Proud Marxist Agitators}
While a full discussion of the Marxist agenda is required, for now suffice that the attitudes Marxist propagators possess are amusingly contradictory.
All these intellectual roustabouts accomplish is to create a distraction for the managerial class.
Marxist and their cult offspring intend on sowing the seeds for a revolution that will never come.
For a god, communism, that never was.
And askew standards of political sensibilities to blacken their souls.
Grievance-mongering is exciting at first but its novelty wears off and pays diminishing returns in the long run.
Running a revolution by pandering to the anti-society crowd is a fool's errand.
All it does is elevate the foul stinking muck in the soul of every person to the vehicle of political action.
But subjectivism as a site of political innovation is a dead end.
And Marxism as a whole has been a festering failure that refuses to know it died a century ago.
Something must replace it with the dual purpose of burrying the corpse of Marxism and scaffolding an alternative route to the utopia.

The more Marxist agitate, the farther away we get from the utopia.
And maybe Marxists know this.
It was damning easy enough for the Marxist to kill and rape in the name of progress and utopia.
One wonders if these sadists have given up pursuit of utopia and now act, lashing out, like some wounded and ultimately surrendered animal.
A time will come for us to write one wicked eulogy for these monsters.
But now is not the time.
For now, Marxist and their mutant bastard children owe everyone an explanation why they negated the utopia.

\subsection{Last Words on Leftism and the Managerial Class}
It should be clear that not every idea that can nominally found on the left is a good idea.
Nominally, we find the managers to be on the left.
But this is not novel revelation.
Any class of people that seek to enhance their own power at the expense of a established and tired one will be on the left.
It is frightening that this must be said but you don't have to die for every poor idea regurgitated by leftist demagogues.
I would you must exercise a degree of intellect when perusing the vast universe of the political left to find the gems in the rough.
But if you find yourself in service of every leftist idea you come across, you probably should not be thinking for yourself, let alone for others.
This point resonates doubly if you find yourself in the friendly company of Marxists.
As with most class of men, the managerial class will be on the left as long as they have more ground to cover for themselves.
As soon as they feel they have exhaustive and secured their own interest, they will become a potent conservative force.
I say that before that incredulous turning point is experienced, it would be preferable that the managerial class be dismantled.
It is not a matter of naive nominal political placement but of a class of men possessing power they did not earn and influence they do not deserve.

\subsection{Unfounding the Managerial Class}
Before I propel myself into the general theory of undoing the managerial class, I would like to address the more immediate concern.
I do not wish to recreate for myself the rapacious errors of Marxists.
By wishing to remove the influence of the managers and perchance dismantle their existence is not a wholesale invitation to wish them evil.
I do not want them hurt.
I do not want to propagandize against them.
I do not wish to groom an impressionable generation of youth to hate them.
I wish to evade the evil that has already been in supposed pursuit of a revolution that will bring about the utopia.
The reason these reservations of mine are necessary is because leftist have taken a nasty habit of leaving their activism unbounded and implicit threat of violence as an acceptable option.
I explicit forbade violence of any sort.
This is not something a Marxist would ever say because their whole program is the subtle flirtation of hybridized political violence.

No such lofty rhetoric is necessary.
I believe my theory is more comprehensive, more benign, more technological than anything Marxists could have summoned.
If you would allow this cranky hermitic theoretician his due, I would like to posit the general outline of a speculative theory.
This theory will not only defenestrate the managerial class but also any other up and coming competitors to the managerial throne.

The idea is that it is not enough to dismantle class.
Naively doing so would create a simple vacuum of power that nature does find contemptible.
I take inspiration from the open source software movement.
Open source software can be construed as a movement that explicitly keeps out corporate interests.
This is done not by design, as open source appears more like a buzzing bazaar than a cohesive political movement.
Thus, the best vaccine against corporate meddling is gathering people that would happily do the work for free given the ability.
I posit something similar for exorcising the spirit of the managerial class.

As I said before, what gives the managerial class its power is the vagueness of the job description.
A more concise explication of the job exists only in the head of the individual manager.
This knowledge is earned by their on the job experience and the amalgamation of general principles synthesized with the former.
If we were to create a system that would challenge the managerial class, it would be at the very least a system that does what the manager does productively.
I author the creation of two systems below to motivate this end.

\subsubsection{The Pedagogical Object}
Managers in practice behave very much like investigative detectives.
They come in after the fact and try to piece together what happened.
It is this piecing together that is the most important part of the managerial job.
There is a manager then for every modality of concern of enterprise.
Finance, marketing, sales, human resources, auditing, etc.
All of these domains and more demand a specific type of manager to oversee them, if not investigate them like the auditor.

One hires a financial auditor after fact to establish facts in the domain of finance.
One hires a lawyer after the fact to establish facts in the domain of law.
And so on.

This mechanism I feel is not by design but by necessity.
A layer is necessary for enterprise but having a lawyer on staff is too expensive.
Thus is this system of investigative managerialism born.
I claim that managers at their respective workplace play a similar role.
This is by design as intrigue at their workplace is their purview.
Only they are empowered to possess the knowledge of the inner workings of the enterprise.
This is a problem and easily leads to a tyranny by manager.

What if analytic objects of managerial concern contained readily resources for the judicious expenditure of their prospective domain?
There would be objects that contain already what is needed to fully understand their inner workings.
This would be a pedagogical object.
We would start with the most basic of objects.
From this, we can scaffold up to an entire enterprise.

For example, instead of hiring a lawyer, the object of interest already possess what is necessary for anyone to act as a lawyer.
Through pedagogical material, the object instructs any user of the object to act in the robust capacity of a lawyer.
Think of the object as imbued with a pedagogical essence that teaches how to consume the object.
For the legal domain, it could be a course on the law.
While this may seem challenging, I feel that at market for these types of objects will provide impetus for innovation in this regard.

Thus is the consumption of the object feasible without resorting to a managerial class.
If everyone can be a manager, then no one is a manager.

\subsubsection{The Transparent Society}
In the same spirit of Popper and the open society, I would like to propose the transparent society.
In this society, there is no privacy because there is no need for privacy.
Nothing in the phenomenology of the subjective experience is deserving of remorse or shame.
This can only occur when the private behavioral conduct of the individual can be said not to jeopardize the integrity of the ruling political regime.
If this sounds utopic it is because I am defining the conditions of the utopia, which also happen to preclude a managerial class.
Of course this sounds utopic.
I do not presume to preclude that easier solutions exists.
Perchance lusting after utopic things is an exercise in futile vanity.
What I find comforting is that utopia can be found in variable abundance in different things.
I truely believe that my pedagogical object is feasible in our current society.
Then by the grace of God, do we hold not in contempt the small utopias we do have while waiting for the full one.

As I have said before, I believe reasonably vaunting towards the utopia is proud endeavor.
A transparent society will open a maddening level of things to public analysis.
That is a good thing.
No longer can managers hide in the shadows.
No longer can power pretend to be benevolent.
What discomfort you might experience with the thought of your trivial deprivations experience in your private life being aired are nothing to those of managers.
What you lose in monopoly of your private life, you gain in the fair landscape that will eviscerate hidden evil.
I believe this to be pivotal in the destruction of the managerial class.
Not just that, I believe all private conspiracies against humanity will be exposed to the excoriating, remorseless tearing of the public.
This is a good thing.

Additionally, any corporate effort, industrial or otherwise, that does survive the light will be forced to justify itself.
This will add a level of accountability and emphasis on general theory.
I believe there will be prodigious innovation in this regard.
This theory can thus form the basis for anyone to justify their actions.
And more importantly, it can be used against the corporate interest, should they not live up to their lofty ambitions.
As a devout and honest member of society, I admit to a fiendish glee at the prospect of society cannibalizing the selfish among us.
I look forward to the prospect of devouring the wicked among us.

\section{Summary}
This paper constitutes my attempt at an invitation for others to learn more about their enslavement to the managerial class.
I have tried to be as comprehensive as possible in my analysis.
At the same time, I know this paper will be found wanting and of limited scope.
I invite readers to undertake their own research and to share their findings with the world.
Only when knowledge is free, will men be free.

\subsection{Causes of the Managerial Class}
\begin{enumerate}
    \item Separation of Ownership and Control of Capital
    \item Unbounded Domain of Managerial Discretion
    \item Ambiguous Job Description
    \item Monopoly of Industrial Knowledge and Insight
\end{enumerate}
\subsection{Consequences of the Managerial Class}
\begin{enumerate}
    \item Outsized Receipt of Wages Demanded by the Managerial Class
    \item Managerial Discretion as a Mechanism for Rent Seeking
    \item Discretion Defining Public Discourse
    \item Distracting the Public from Their Outsized Influence
\end{enumerate}
\subsection{Destroying the Managerial Class}
\begin{enumerate}
    \item Raising Public Awareness of the Managerial Class
    \item Creating Consumable Pedagogical Objects
    \item Transparent Society
\end{enumerate}

\section{Suggested Points of Departure for Future Research}
While the utopia is goal worth striving for (if you're a leftist), I believe there to exist solutions that may not require it.
The reason this is important is because utopic striving has been given a bad name by all the failed attempts at it.
This is a natural outcome of failed utopic experiments that take a rapacious character.
All failed utopias will push the pendulum of public opinion towards the opposite extreme.
That is by design and cannot be avoided.
The general character of a utopia will be free, voluntary, non-coercive and out-compete the current regime.
Anything that falls foul of any of these criteria will be doomed to failure and will only delay the utopia.
The utopia is in the making of the utopia.

Given all that, dethroning the managers should not be an exercise in leftism.
Including the non-left would be a boon to the destruction of this class.
I believe it to be important enough to warrant the inclusion of all political persuasions.
Expulsion of the right only creates fascists, as history has shown.
If any political gambit is to succeed, everyone must be shown that they possess a stake in the proceedings, not just the left.
Thus, I believe a subgoal of this venture should be whether a non-leftist orientation can be built upon.

\end{document}
