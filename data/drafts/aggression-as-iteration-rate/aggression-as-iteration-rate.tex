\documentclass[12pt]{article}
\usepackage{amsmath}
\usepackage{amssymb}
\usepackage[utf8]{inputenc}
\usepackage[margin=1in]{geometry}

\title{
Aggression as Behavioral Iteration \\
\large A Comprehensive Theory of Domain Specific Aggression }
\author{Salvador Guzm\'an Jr.}
\date{June 29, 2023}

\begin{document}

\maketitle
\pagebreak

\begin{abstract}
    Long has behavior been studied as a indeterministic process.
    I believe this to be not for simple lack of theoretical underpinning.
    A certain degree of Romantic affection has blinded our analytical scrutiny to pursuing such an end.
    Given that I am properly addressing this established circumstance, I wish to remedy this condition.
    Whatever resonance once possess for the human condition, it should not distract from scientific and analytic regor.
    With the Romantic vision of humanity discarded, we can understand behavior and behavior changes as a simple Mathematical relationship.
    This paper will attempt to establish a mathematical theory for behavioral iteration.
    The mathematical theory underpinning human behavior and all animal behavior rests on well-understood axioms of measure theory and probability theory.
    With this in mind, I will attempt to establish a mathematical theory for behavioral iteration.
\end{abstract}
\pagebreak
\tableofcontents
\pagebreak

\section{Why not A Mathematical Theory for Behavioral}
This author cannot help but be impressed at the scant attention paid behavior.
It is often the sensibilities of our times that establish the topologies of allowable study.
The study of human behavior is a topic that is of great interest to many but no particular interest to scientist.
I venture the position that this is because the study of humanity proper as a biological species offers to intimate a view of the human condition.
There are things we often would care to omit from our understanding to protect our fragile view of the world.
This author admits to no exceptions in their own case.
However, I believe that intellectual maturation necessitates examining the world as it is, not as we would like it to be.
And if one were to seek to change it for the better, one better have accumulated a formidable body of knowledge first.
As an intellectual, I believe that two great sins any thought-minded person can commit is changing what they don't understand.
The second is refusing to countenance aspects of reality they find disagreeable and preventing others from doing the same.

With all that said, we can begin on our ambitious adventure to describe behavior as a mathematical process.

\section{A Mathematical Theory for Behavioral}
The idea that human behavior can be understood as a mathematical process is not novel.
Yet it stands as one of few pillars of modern life that has resisted attempts to standardize and formalize.
I feel this is more due to our naive sensibilities than any particular theoretical hurdle.
Perchance there exists room for doubt that such a thing was possible until a couple of centuries ago.
Before probability theory and measure theory, it would have been very difficult for such an advent to even start.
Given that we have been met with these elegant tools, no such excuse is possible now.

As I have intuited before, I associate the relative naive state of this field of study to sensibilities
The prodigious explosion of theoretical tools produced in the last two centuries and their judicious application to domain problems testifies to their success.
That they have not ventured to a sensitive topic and rendered out God from the process says more about us than our tools.
I believe that the time has come to remedy this situation.

What I propose is a mathematical theory for behavioral iteration.
I wish to put forward a theory that can describe behavior to a certain degree of certitude given the resolution of our tools of analysis.
I believe that human behavior can be understood as the sum of sources of behavior.
What this entails is that there biological sources of behavior that ebb and flow in strength and frequency.
The environment and biological cues are what decide the strength of the call for specific behaviors.
Thus, one can see the body as a holding cell for a multitude of behavioral impulses.
These impulses "fight" for supremacy and monopoly over the modalities of action.
That is, because of the scant morphology of the human body, mental focus cannot be so easily divided.
Thus, the body is a battleground for the impulses of behavior, each vying for monopoly.

\subsection{An Expedition into Behavior}

The relative strength is affected by multiple factors. The most noteworthy are,
\begin{enumerate}
    \item Time of day (for diurnal animals)
    \item Relative biological premium (as relating to homeostasis)
    \item Bounds of modality (lower modality behaviors are more less likely to experience competition)
\end{enumerate}

\subsubsection{Sleep}
A few words on the above.
The time of day is a factor for animals that experience sleep.
Sleep is time-sensitive in diurnal species and thus the relative strength is controlled by time-sensitive factors.
In the case of humans, the circadian rhythm is the most important factor.
It incorporates the presence of light, which is assumed to be sun-derived.
The presence of melatonin moderates the strength of the sleep signal.
The relative biological premium is a factor that is related to the homeostatic state of the organism.

\subsubsection{Biological Premium}
As the last sentence alludes, some behaviors have a biological premium associated with their activation.
Food is the most obvious example.
It is a metabolic imperative that an organism satisfy their nutritional requirement to continue living.
This is not negotiable.
And thus, the impulse strength to consume food will be very high.
Of course, that does not mean the strength of this signal will not be the same over all conditions.
When an animal is near starvation, the signal will be very strong.
When an animal is satiated, the signal will be very weak.
Thus, food and consumption thereof is the sink that the drive to eat advances the creature towards.

The satiation of this signal can be seen as the level of scarcity of food.
The more abundant food is, the more likely the animal is to consume it and stay satiated.
The more scarce food is, the more likely the creature will expend energy in search of food, under the influence of the drive to eat.
The satiation of the drive can be seen as an external event that triggers a fall in the relative strength of the drive to eat.
In the calculus of power, acceding this particular behavior leads to competition among the remaining impulsive power-brokers.

\subsubsection{Morphological Modality}
The morphology of an organism is necessarily limited.
In this discussion, morphology does not just refer to the physical shape of the creature.
It will also refer to the dedicated manner in which the creature achieves homeostasis.
For example, humans are bipedal and thus have access to hands while standing.
This is a modality that most primates do not have access to.
Impulsive and behavioral access to what the hands are doing is a modality that humans will experience.
The same can be said for anything that is available in relative abundance.
Attention is also a modality that is available in limited supply.
We can talk about splitting focus happening at the expense of the quality of the dedicate focus.

\subsection{Abstract Discussion}
As of now, I have ambiguous whether I am referring explicitly and narrowly to humans.
I wish to continue speaking in this manner.
Note, that what I am stating is general and applies to any organism.
Obvious, the specific case of humans commands our attention.
Just keep in mind the general nature of the statements.
The mathematical machinery is rather abstract and general by design.
Thus, do I hope the reader forgives any trepidation experienced when applying this model to the non-human case.
In the future, discourse will have to cater specifically to other animals.

\section{What is Behavioral?}
\section{What is Aggression?}
\subsection{Aggression as a Strategy}
\subsection{Aggression Tolerance as Contingent on Personality}

\section{What is a Strategy?}
\subsection{Strategies For an Iterated Game}

\section{Aggression as Pursuit of Winning Strategy}

\section{Summary}
This paper constitutes my attempt at an invitation for others to learn more about their enslavement to the managerial class.
I have tried to be as comprehensive as possible in my analysis.
At the same time, I know this paper will be found wanting and of limited scope.
I invite readers to undertake their own research and to share their findings with the world.
Only when knowledge is free, will men be free.

\subsection{Behavior as a Mathematical Process}
\begin{enumerate}
    \item 
\end{enumerate}
\subsection{Aggression as Strategic Iteration}
\begin{enumerate}
    \item 
\end{enumerate}

\section{Suggested Points of Departure for Future Research}

\end{document}
