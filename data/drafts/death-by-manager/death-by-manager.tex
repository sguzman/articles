\documentclass[12pt]{article}
\usepackage{amsmath}
\usepackage{amssymb}
\usepackage[utf8]{inputenc}
\usepackage[margin=1in]{geometry}

\title{
 Death By Manager \\
 \large The Mangerial Class and the Exploitation of the World}
\author{Salvador Guzman}
\date{June 24, 2023}

\begin{document}

\maketitle
\pagebreak

\begin{abstract}
    While popular discourse has focused on the role of the capitalist class in the prolonging of certain deprivations, the managerial class has escaped all culpability.
    This paper seeks to rectify this oversight by examining the rapacious tendencies of the managerial class and its role in the exploitation of the world.
    Exploitation is interpreted rather conservatively in this paper.
    Frankly, it is not needed.
    There will be no resorting to Marxist definitions to extract sensible attitudes to disposition one against the managerial class. Instead, the paper will focus on the managerial class, what they are, how they came to be, the consequences of their unattenuated power, and how to dispose oneself against them.
    You will find no panacea here; only sagacity and wisdom of the analytical word. What motivates this author to write this paper is the envisioning of a better world beyond the managerial horizon.
    A world where the resources of the world are not held hostage by a hostile, callous and neglectful interest.
    This can ours, reader.
\end{abstract}
\pagebreak
\tableofcontents
\pagebreak

\section{The Manager as the Apex Corporate Predator}
A tragedy in the making can be spotted in the manner by which private interests can define the contours of popular discussions.
While the modern imagination is still hauntec by the trauma of the 20th century, there lurks a class of men and women, more detestable by action, than communists or capitalists.
I am of course referring to the managerial class.
Whatever be the country's specific ideological affiliation, it produces a niche for the same group of men that hazard an opportunity to sate their ambition.
While the particular job description of the particular manager may vary, what does not vary is their outsized influence.
This influence extends beyond the narrow confines of the workplace and bleeds into the public sphere.
And bleed indeed as the managerial class has shown themselves to be a rapacious bunch.
In exercise of their power and appraisal of their own interest, they have consumed the world like no class before them.
To safeguard their longevity, they have erected a predatory system of control that is as insidious as it is pervasive.
No industry is safe from their influence nor extends beyond their narrow discretion.
Thus, it is the narrow description of their particular profession, instantiated many times over, that lead to their proliferation.
It is the sincere opinion of this observer that the dysfunctional state of political course is a product of their outsized influenced.
As a class, they respond to local and greedy incentives that maximize their own utility.
And because of the unbounded nature of corporate power, does the tendrils of their influence extend to the world.
It is the purpose of this paper to examine the managerial class, their role in the exploitation of the world, and how to dispose oneself against them.

\subsection{Capitalism vs. Communism: The Perfect Distraction}
The 20th century was a century of ideological conflict.
The two dominant ideologies of the time were capitalism and communism.
While this is a gross butchery of the conflict, it serves well our cursory glance.
What we become interested in is the managerial experience and conduct in both ideologies.
\subsubsection{Capitalism}
In capitalism, the managerial class is a product of the separation of ownership and control.
The separation of ownership and control is a product of the industrial revolution.
A more detailed discussion on the cause of the managerial class within a capitalism awaits us.
For now, we must understand that the separation of ownership and control was necessitated by the relative complexity of industrial enterprise.
In this paper, capitalist is understood as the owner of capital and nothing more.
No deragatory connotation is implied.
No longer could a single capitalist manage the affairs of the enterprise.
Rather, they were forced to delegate away the responsibliity of management to another group of people.
The more the capitalist owned, the more the daunting task of management became reality.
It was simply not feasible to supervise all of productive work of their capital.
As a class of people, capitalists choice to delegate to a proxy the problem of management.
This event is what dawned the separation of ownership and control.
And in this separation do we find the genesis of the managerial class.
Because of the unbounded nature of management, the managerial class was allowed to grow, often at the expense of other workers.
They became de facto owners of the enterprise by virtue of the power they exercised.
And it is in this power that they reveled.

\subsubsection{Communism}
The managerial experience differed very little in communism.
While certaintly the profit motive shrank in importance, the need to manage large enterprise did not.
As can be observed in the many cases of practiced communism, the managerial class was well regarded.
Large enterprise would always be the bedrock of their necessity and in the absense of capitalist, their respective power could only grow.
To say that local politics of day to day management are what shaped the manager's intuition is a gross understatement.
In fact, competition between managers was the norm.
And competition between different regions characterized the political landscape.
A common trend you will observe is that in competing with each other, they inadvertently expand their domain.
This is a common theme in the managerial experience due to the unbounded nature of management.

Do note that lackluster analysis in this section.
That is by intention.
This paper is not about communism.




\section{Motivating Musings over the Managerial Experience}
But why muse over the managerial class?
Cer

\subsection{Rise of the Managerial Class}
\subsubsection{Separation of Ownership and Control}
\subsubsection{Introduction of Managerial Discretion}
\subsubsection{Death of the Profit Motive}

\section{The Exploitation of the World}
\section{Rise of the Petty Tyrant}
\section{Tyrant by Discretion}

\section{Suggested Points of Departure for Future Research}


\begin{enumerate}
    \item ...
\end{enumerate}
\end{document}
