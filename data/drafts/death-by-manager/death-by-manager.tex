\documentclass[12pt]{article}
\usepackage{amsmath}
\usepackage{amssymb}
\usepackage[utf8]{inputenc}
\usepackage[margin=1in]{geometry}

\title{
 Death By Manager \\
 \large The Mangerial Class and the Exploitation of the World}
\author{Salvador Guzm\'an Jr.}
\date{June 24, 2023}

\begin{document}

\maketitle
\pagebreak

\begin{abstract}
    While popular discourse has focused on the role of the capitalist class in the prolonging of certain deprivations, the managerial class has escaped all culpability.
    This paper seeks to rectify this oversight by examining the rapacious tendencies of the managerial class and its role in the exploitation of the world.
    Exploitation is interpreted rather conservatively in this paper.
    Frankly, it is not needed.
    There will be no resorting to Marxist definitions to extract sensible attitudes to disposition one against the managerial class. Instead, the paper will focus on the managerial class, what they are, how they came to be, the consequences of their unattenuated power, and how to dispose oneself against them.
    You will find no panacea here; only sagacity and wisdom of the analytical word. What motivates this author to write this paper is the envisioning of a better world beyond the managerial horizon.
    A world where the resources of the world are not held hostage by a hostile, callous and neglectful interest.
    This can ours, reader.
\end{abstract}
\pagebreak
\tableofcontents
\pagebreak

\section{The Manager as the Apex Corporate Predator}
A tragedy in the making can be spotted in the manner by which private interests can define the contours of popular discussions.
While the modern imagination is still hauntec by the trauma of the 20th century, there lurks a class of men and women, more detestable by action, than communists or capitalists.
I am of course referring to the managerial class.
Whatever be the country's specific ideological affiliation, it produces a niche for the same group of men that hazard an opportunity to sate their ambition.
While the particular job description of the particular manager may vary, what does not vary is their outsized influence.
This influence extends beyond the narrow confines of the workplace and bleeds into the public sphere.
And bleed indeed as the managerial class has shown themselves to be a rapacious bunch.
In exercise of their power and appraisal of their own interest, they have consumed the world like no class before them.
To safeguard their longevity, they have erected a predatory system of control that is as insidious as it is pervasive.
No industry is safe from their influence nor extends beyond their narrow discretion.
Thus, it is the narrow description of their particular profession, instantiated many times over, that lead to their proliferation.
It is the sincere opinion of this observer that the dysfunctional state of political course is a product of their outsized influenced.
As a class, they respond to local and greedy incentives that maximize their own utility.
And because of the unbounded nature of corporate power, does the tendrils of their influence extend to the world.
It is the purpose of this paper to examine the managerial class, their role in the exploitation of the world, and how to dispose oneself against them.

\subsection{Capitalism vs. Communism: The Perfect Distraction}
The 20th century was a century of ideological conflict.
The two dominant ideologies of the time were capitalism and communism.
While this is a gross butchery of the conflict, it serves well our cursory glance.
What we become interested in is the managerial experience and conduct in both ideologies.
\subsubsection{Capitalism}
In capitalism, the managerial class is a product of the separation of ownership and control.
The separation of ownership and control is a product of the industrial revolution.
A more detailed discussion on the cause of the managerial class within a capitalism awaits us.
For now, we must understand that the separation of ownership and control was necessitated by the relative complexity of industrial enterprise.
In this paper, capitalist is understood as the owner of capital and nothing more.
No deragatory connotation is implied.
No longer could a single capitalist manage the affairs of the enterprise.
Rather, they were forced to delegate away the responsibliity of management to another group of people.
The more the capitalist owned, the more the daunting task of management became reality.
It was simply not feasible to supervise all of productive work of their capital.
As a class of people, capitalists choice to delegate to a proxy the problem of management.
This event is what dawned the separation of ownership and control.
And in this separation do we find the genesis of the managerial class.
Because of the unbounded nature of management, the managerial class was allowed to grow, often at the expense of other workers.
They became de facto owners of the enterprise by virtue of the power they exercised.
And it is in this power that they reveled.

\subsubsection{Communism}
The managerial experience differed very little in communism.
While certaintly the profit motive shrank in importance, the need to manage large enterprise did not.
As can be observed in the many cases of practiced communism, the managerial class was well regarded.
Large enterprise would always be the bedrock of their necessity and in the absense of capitalist, their respective power could only grow.
To say that local politics of day to day management are what shaped the manager's intuition is a gross understatement.
In fact, competition between managers was the norm.
And competition between different regions characterized the political landscape.
A common trend you will observe is that in competing with each other, they inadvertently expand their domain.
This is a common theme in the managerial experience due to the unbounded nature of management.
Thus is the vampire of the managerial class able to transcend their wicked fangs over all of humanity.

Do note that lackluster analysis in this section.
That is by intention.
This paper is not about communism.

\subsubsection{Fascism}
Fascism is a managerial ideology.
That should be self-evident from common parlance related to fascism.
What should also lead one to suspect the manegerial nature of fascism is the emphasis it places on the management experience.
The way a manager runs their respective workplace is like the perfect fascist society.
The manager is the dictator of the workplace.
This should not in any way be a surprising conclusion.
The common experience that many Americans workers have with their respective managers is one of a petty tyrant.
The experience is commonly negative because of the unbounded nature of management.
While Nazi Germany will evade our analysis, proper examinations of the role of managers in industry and politics exist.
The reader is encouraged to read these and see how the death of the profit motive, which also existed in Fascism, did not preclude managers.
Rather, it was the perfect environment for them to thrive.
They are the peak corporate parasite of all industry.


\section{Motivating Musings over the Managerial Experience}
But why muse over the managerial class?
Certaintly, would an exhaustive analysis of the managerial class reveal anything novel?
I think so.
Beyond such a speculative impetus, analysis of the managerial class has suffered from lack of development compared to other institutions.
This is evident from the lack of public consciousness over managers as a class and commensurate understanding of their influence.
Common discussion that prevail tend to overemphasis the role of the capitalists and diminish the role of managers.
While it is natural to fail to appreciate this role as most people's understanding of a manager is exhauted by their respective profession, it does not tell the full story.
Surely, the manager class has partipated in shaping the contours of the public consciousness.
And in this shaping, they chosen, whether intimately conscious of their endeavor or not, to obscure their own role.
Their obscurity is well known.
One imagines a similar situation occurred in the late 19th century with the rise of the capitalist class.
The rise of a new class of power brokers is always traumatic to those they discolate.
And the managers are no different.
The capitalists would at times spend large sums of money to propagandize their own role and rehabilitate their image in the public eye.
Preferred to this however would be obscurity.
Ultimately, the capitalists have no convincing argument motivating why their power should continue to accrue unabated.
And such is the situation that we find with the managerial class.

While the capitalist class has been well studied, the managerial class has not.
This is by design.
The capitalist class was eventually forced by circumstanced to reveal themselves, their position, their power and their ideology.
The managerial class has not faced any sort of reckoning yet.
I must remind the reader that the promulgation of the capitalist view of the world needed to be cajoled out of their exponents.
Such theory was not forthcoming without bringing the conflict to the class proper.
And will the managerial class be any different?
No.
It is this fight that I wish to prepare you for, dear reader.

\subsection{Rise of the Managerial Class}
The previous explication should have established the genesis of the managerial class.
More important, with any new class of people that ultimately define the contours of power, the cause of their rise is of paramount importance.
The managers did not create themselves, ex nihilo.
The social conditions that preceded their rise are what ultimately ensconced a world with managers.
Once a critical mass of managers were in existence, the tired and pervasive laws of power took over.
Given the unbounded nature of management, from the small initial cliche of managers instantiated across industry came the rise of the managerial class.
From there, the algebraic laws that constitute the calculus of power demanded that managers be delegated more and more responsibliity as industry and enterprised matured.

Let us examine some of the aggravating factors that led to the rise of the managerial class.

\subsubsection{Separation of Ownership and Control}
With the rise of industry, the laborial demands placed on the capitalist class grew.
No longer could a single capitalist manage the affairs of the enterprise.
Industrial enterprise had reached a point of maturation where the task of management was too daunting for a single person.
Thus, the capitalist was not able to be a capitalist and a manager.
One of the tasks had to be delegated.
And it was the task of management that was delegated.
By virtue of this arrangement, the capitalist could maintain their ownership of the enterprise while delegating the task of management to another group of people.
And here do we find the genesis of the managerial class.
In preferring to be a capitalist (to retain their commensurate power), they chose to create a class of people.
This class of people would be the managers.
This simple act not only authored into existence a class of men but also separated spheres of concerns.
It is this divorce that created a role for a power-conscious elite to ultimately capture.
And capture they did.

The ultimately motivating factor for the capitalists was the desire to retain their power.
This by itself is not a novel observation.
All creatures, simple or complex, seek to retain their power.
What makes the case of the capitalist unique is that the rational calculus of power demanded that they create a new class of people.
To retain their power, they needed to dissolve the constraints faced by enterprise.
They could not feasible be managers and capitalists.
The demands on their time would be too great for any person.
On a more fundamental level, the capitalist class would not be able to enjoy the fruits of their labor.
It is not that capitalists abhorred work.
They were happy to work to build up the incipient level of complexity in the first place.
In their delegation, they shirked laborial duties and happily retained ownership.
They viewed their ownership as safeguarding their power even though a proxy would run their enterprise.
And should any conflict arise, they esteemed that their de jure ownership would be suffice to reconcile any conflict with management in their favor.

Of course, the capitalist did not envision that their rational and locally-greedy choices would create a new class of men.
They did not envision that they would lose influence over the affairs of their enterprise.
Again, their ownership was supposed to be a safeguard against this.
However, because of the calculus of power and the unbounded nature of management, the capitalist class was ultimately dislodged from their position of power.
The managerial class would ultimately capture the enterprise over decades as more and more of industry was "managed."
The capitalist class experienced a "hollowing-out" of their influence as the managerial class grew in power at their expense.
This happened because of local decisions.
The capitalist class did not envision that their local decisions would have global consequences.
In the end, the opportunities that enticed managers into existence offered them opportunities to exercise their volition.
By virtue of management, their volition was tied to management.
And by managing industrial enterprise, they were able to exercise discretion to their own benefit.
Again, by virtue of managing, the will of the manager is interpreted as the will of the enterprise.
And thus was managerial discretion the helm of industry.

\subsubsection{Managerial Discretion as the Rapacious Tyrant}
A beguiling question is why did not the capitalist class see the possible hazard of managerial discretion.
That their own discretion would be usurped in the process of delegation surely was a possible contingency.
And that the enterprise would succumb to managerial capture did not evade their notice.
The profile inviting analysis of the manager is even more perverse than how the situation started.
It is 2023 and even now, the discretion of the managers has not been fully realized by the public, let alone critiqued.

Owner or not, safeguard of the capitalist's de jure ownership was never in jeopardy.
Perhaps the capitalists had a blind spot that emphasized the legal fiction that protected their property rights.
Perhaps they did not value de facto discretion if they could ultimately resolve any conflict by resorting to their legal narratives.
Whatever the reason, it can be understood that the capitalist either overplayed their legal obligations or did not value discretion as practiced.
Slowly the effect that the capitalist had as a manager eroded because they let it erode.
And filling the vacuum became a natural proclivity of the new class of managers.
That they greedily feasted on the opportunity to become intoxicated with their own discretion is not the story.
The novelty is that the same local decisions made by capitalist all around the world but mostly in countries with advanced enterprises lead to the instantiation of this class.
There was no coordination between the capitalists.
Their own greed was the guiding light that rowed them to their own perdition.


\subsubsection{Death of the Profit Motive}
Rationality and the power motive succumb to the same calculus.
That some rigorous theory undergirds the exercise of both should not startle any intelligent person.
All things are knowable and all things are predictable in the abstract with a robust enough theory.
And power, money and industry are no different.
Celestial mechanics and the wonderful creatures that inhabit it are no strangers to the terristrial firmament that binds their calculus.
And even God must succumb to theory.

If some adversarial party were to attempt at elucidation of the calculus at play of the managers, what would they have to defer to?
What externally accessible evidence exists to support the existence of this managerial class?
Aside from the astute socially available observations that testify to a like-minded and interested party, we have the death of the profit motive.
The profit motive is the most salient and obvious evidence of the existence of industry and capital at work.
Naive intution would inform us that profit is necessitated by avaricious churning of the capitalists.
Without being able to realize a return on their investment, folklore states that the capitalist would dissuade exposure to their capital.
And yet, the profit motive is dead.
How do we reconcile this?
Did the capitalists turn into philanthropists?
Do they donate their capital out of the kindness of their heart?
No.

The profit motive is dead because the managers have no direct incentive to realize a profit.
A prolific incantation of their professional creed involves them managing.
Nothing else.
The hearty implication is that they will manage in proxy of the capitalists and that they will manage in the best interest of the capitalists.
And if that not realized, they can be removed from their position by the latter.
In absence of heavy handed intervention, they will manage how they see fit.
Any attempt by the capitalist to infuse the enterprise with their own discretion will be interpreted as malicious micromanaging.
In fact, the word micromanaging is quite telling since the managers proper consider interference in their jurisdiction as being beset by a petty "little manager."
It is assume by allusion to their respective job title that the manager rules in all things management.
Hence it being no little understatement that the manager must be a fascistic dictator.

While a singular manager can be fired, a class as whole is not so easily removed from their volition.
Managers will manage, regardless of their individual idiosyncrasies.
And maybe if the capitalist, the owner of capital, were the sole interest that the management deferred to, firing would stem the managerial tide.
However, a modern enterprise, with its customers, employees, suppliers, board of directors, and shareholders, contains diffuse interests.
The capitalist is no longer alone.
This was the faustian bargain that the robber barons of old contended with in order to expand the domain of their industrial undertaking.
Choose to retain full voting rights of their enterprise and limit their growth.
Or expand their industry and dilute their influence.
The latter was chosen often enough for it to be the norm.

In a diffuse settings, the appearance of serving the interests of the capitalist is sufficient.
To the extent that they accomplish this only goes toward establishing their competence.
Only outright malicious or incompetence ends the reign of a particular manager.
But management as a theory of industrial praxis never recedes from this earth.
The profession category is never eroded.

Thus, given that all has been said, where is the profit motive?
It appears peculiar that any particular occupation may escape the profit motive.
But if there were to be one, it would be management.
Management is the most removed from industrial output.
They do not directly contribute to production quota but by virtue that the sites of production continue unemcumbered.
Anything above outright atrocious failure is success for the manager.
Hence they are protected from having to manage a profit into existence.
Profit is not relegated to their purview but to the capitalist.
Any profit realized thus is a boon to their credit.
If the manager does not realize a profit, they are penalized because it was never their responsibility to do so.
While this may seem innocuous in the local sense, at large this is the knife in the back for felling the managerial class.

\subsubsection{Managerial Job Security}
Speaking of felling, who manages the managers?
Another manager of course.
Marvelous.
Enterprise as a hierarchy of managers is management run.
And given that all entrepreneurial type of people are extremely conformists, eventually management bleeds into everywhere.
Outside from the direct stakeholders of any enterprise, managers are allowed to act with near impunity.
In fact, the only people managers have to answer to are other managers.
The managerial advantage in job security is real.
There is even a practice of measuring the competency of employees called annual reviews.
Who by sheer chance do you think performs these reviews?
The managers of course.
Do employees underneath a manager have any say in the review of their manager?
Do employees get to review their manager in turn?
No.
And when the manager proper is reviewed, who do you think performs the review?
Another manager higher up in the hierarchy.
It's managers all the way up.
This is the rule by manager alluded to earlier.

\section{The Exploitation of the World}
\section{Rise of the Petty Tyrant}
\section{Tyrant by Discretion}

\section{Summary}
\subsection{Causes of the Managerial Class}
\subsection{Consequences of the Managerial Class}
\subsection{Destroying the Managerial Class}

\section{Suggested Points of Departure for Future Research}


\begin{enumerate}
    \item ...
\end{enumerate}
\end{document}
