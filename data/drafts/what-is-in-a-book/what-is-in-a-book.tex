\documentclass[12pt]{article}

\usepackage{amsmath}
\usepackage{amssymb}
\usepackage{amsthm}
\usepackage{graphicx}
\usepackage{hyperref}
\usepackage{color}
\usepackage{enumerate}

\begin{document}
\title{What is a book?\\
\large Fungibility of Books and Their Composite Words}
\author{Salvador Guzman}
\date{\today}

\maketitle
\newpage

\begin{abstract}
Books capture the essence of knowledge distilled into a verbal consolidation. In
this sense the physical medium that constitutes the books proper is not what
compels attention. What truly continues to rapture our imagination is the
stability of the written word. Written documents have proven that communication
can be stored, transmitted, reproduced and expressed in a variety of ways. All
wealth of actions that the written word commands can be done all without every
modifying the original intent. This is the essence of the word and the
books that house them. This can only be done by accepting certain fungible
facets of words, books and knowledge. A word is a word and knowledge can be
built incrementally without regard to any purported \textbf{sine qua non} that may
lurk in the shadows. In this document, I make the case for the fungibility of
the written word.
\end{abstract}

\newpage

\tableofcontents
\newpage

\section{Introduction}
The story of the word is written long and deep in the history of mankind. Our
fascination with what is written demands solid analysis of its causes and its
consequences. For now, I seek to set the context of what is at stake. While I
demure from comphrensive deliberation, I will attempt to establish common cause
with the earnest reader. My thesis statement is that books are fungible. This
hypothesis relies on the premise of the cultural character of the written word.
The best that an author can do is point towards objective truths beyond their
the shores of their time and place. Their epistemological island is of course an
invitation to the reader to explore what the writer could not. The language the
author employs, with its compositional lexical, syntax and semantic
rapprochement are an attempt to scaffold at truths with provincial tools. It the

\section{The Nature of Books}
\subsection{Unique Identity of Books by Way of Typed Tuple}

\section{The Nature of Words}
\subsection{Lexical Composite}
\subsection{Syntactic Structures}
\subsection{Semantic Ends}
\subsubsection{What We Mean By Words}

\section{Ambiguity, the Semantic Devil}
\subsection{Formal Ambiguity}

\section{Intersection of Knowledge}

\section{Semantics as the Fungible Scaffold}

\section{Means by Any Other Mean}

\section{Towards an Algebra of Epistemology}

\end{document}